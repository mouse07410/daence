\documentclass[draft]{article}

\usepackage{amsfonts}
\usepackage{amsmath}
\usepackage{amsthm}
\usepackage{booktabs}
\usepackage[T1]{fontenc}
\usepackage[final=true,backref=page,colorlinks=true]{hyperref}[2016/05/21]
\usepackage{ifdraft}
\usepackage[utf8]{inputenc}
\usepackage[final]{listings}
\usepackage[final]{microtype}
\usepackage[numbers]{natbib}
\usepackage{xcolor}

% -*- Mode: LaTeX -*-

% Colorblind-friendly palette from <https://jfly.uni-koeln.de/color/>.
\definecolor{Vermillion}        {cmy}{0,    0.8, 1}
\definecolor{Orange}            {cmy}{0,    0.5, 1}
\definecolor{SkyBlue}           {cmy}{0.8,  0,   0}
\definecolor{BluishGreen}       {cmy}{0.97, 0,   0.75}
\definecolor{Yellow}            {cmy}{0.1,  0.5, 0.9}
\definecolor{Blue}              {cmy}{1,    0.5, 0}
\definecolor{ReddishPurple}     {cmy}{0.1,  0.7, 0}


\title{\Daence/:
  \Salsa20/ and \ChaCha/ in Deterministic Authenticated Encryption \\
  with no noNCEnse}
\author{Taylor `Riastradh' Campbell
  \email{campbell+daence@mumble.net}}

\bibliographystyle{plainnat}
\renewcommand*{\backrefalt}[4]{%
  \ifcase#1% no citations
  \or% one citation
    Cited on p.~#2.%
  \else% multiple citations
    Cited on pp.~#2.%
  \fi%
}
\DeclareUrlCommand\doi{%
  \def\UrlLeft##1\UrlRight{DOI~\href{https://doi.org/##1}{##1}}%
  \urlstyle{rm}}
\DeclareUrlCommand\url{%
  \def\UrlLeft##1\UrlRight{\href{##1}{##1}}%
  \urlstyle{tt}}
\DeclareUrlCommand\email{%
  \def\UrlLeft##1\UrlRight{\langle\href{mailto:##1}{##1}\rangle}%
  \urlstyle{tt}}

\newtheorem{definition}{Definition}
\newtheorem{lemma}{Lemma}
\newtheorem{theorem}{Theorem}

\newcommand{\lemmaautorefname}{Lemma}

\newcommand{\term}[1]{\textbf{#1}}

% Kludge for bold italic numerals.  Numerals turn out to be in the same
% place in OML encoding as they are in T1 encoding, so even though T1
% is not normally fit for math, it happens to work out here.
\DeclareMathAlphabet{\mathbfit}{T1}{lmr}{bx}{it}

\newcommand{\warning}[1]{%
  \ifdraft{\textbf{\textcolor{red}{\textsc{Warning}: #1}}}
          {}}

\newcommand{\margintodo}[1]{\ifdraft{%
  \leavevmode\marginpar{\leavevmode%
    \parskip=4pt plus 2pt minus 1pt\relax%
    \raggedright%
    \color{red}\bfseries%
    TODO: #1%
  }%
}{}}

\DeclareRobustCommand{\operatorsc}[1]{{%
  \ifmmode\let\next=\operatorname\else\let\next=\relax\fi\next{\textsc{#1}}}}
\def\Salsa#1/{\operatorsc{Salsa#1}}
\def\ChaCha#1/{\operatorsc{ChaCha#1}}
\def\HChaCha#1/{\operatorsc{HChaCha#1}}
\def\HSalsa#1/{\operatorsc{HSalsa#1}}
\def\XChaCha#1/{\operatorsc{XChaCha#1}}
\def\HXChaCha#1/{\operatorsc{HXChaCha#1}}
\def\XSalsa#1/{\operatorsc{XSalsa#1}}
\def\HXSalsa#1/{\operatorsc{HXSalsa#1}}
\def\Poly#1/{\operatorsc{Poly#1}}
\def\Auth#1/{\operatorsc{Auth#1}}
\def\Daence/{\operatorsc{Daence}}
\def\Deuce/{\operatorsc{Deuce}}
\def\NH/{NH}

\def\CAESAR/{\textsc{Caesar}}
\def\eSTREAM/{eSTREAM}
\def\OCB/{OCB}
\def\NaCl/{NaCl}
\def\TweetNaCl/{TweetNaCl}
\def\IETF/{IETF}
\def\NIST/{NIST}
\def\TLS/{TLS}
\def\IV/{\textsc{iv}}
\def\CPU/{\textsc{cpu}}
\def\SUPERCOP/{\textsc{supercop}}
\def\Tarsnap/{Tarsnap}
\def\GHASH/{\textsc{ghash}}
\def\POLYVAL/{\textsc{polyval}}
\def\GCM/{\textsc{gcm}}
\def\GCMSIV/{\textsc{gcm\nobreakdash-siv}}
\def\AES/{AES}
\def\AESCTR/{AES\nobreakdash-CTR}
\def\AESSIV/{AES\nobreakdash-CTR}
\def\AESCMAC/{AES\nobreakdash-CMAC}
\def\AESPMAC/{AES\nobreakdash-PMAC}
\def\AESSIV/{AES\nobreakdash-SIV}
\def\AESGCM/{AES\nobreakdash-GCM}
\def\AESGCMSIV/{AES\nobreakdash-GCM\nobreakdash-SIV}
\def\AESPMACSIV/{AES\nobreakdash-PMAC\nobreakdash-SIV}
\def\naclsecretbox/{crypto\_secretbox\_xsalsa20poly1305}
\def\naclonetimeauth/{crypto\_onetimeauth\_poly1305}
\def\libsodium/{libsodium}
\def\sodiumaead/{crypto\_aead}
\def\OpenSSL/{OpenSSL}
\def\BearSSL/{BearSSL}
\def\libgcrypt/{libgcrypt}
\def\Nettle/{Nettle}
\def\Rust/{Rust}
\def\Go/{Go}
\def\Python/{Python}

\makeatletter
\def\newlatinabbr#1#2{\def#1{\protect\@latinabbr{#2}}}
\def\@latinabbr#1{\@ifnextchar.{\@latinabbrfinal{#1}}{\@latinabbrmedial{#1}}}
\def\@latinabbrfinal#1{\latin{#1}\@}
\def\@latinabbrmedial#1{\latin{#1.}}
\makeatother
\newlatinabbr\eg{e.g}
\newlatinabbr\etc{etc}
\newlatinabbr\etseq{et seq}
\newlatinabbr\ie{i.e}
\newlatinabbr\vs{vs}
\def\latin#1{\emph{#1}}

\def\newmathcronym#1#2{%
  \DeclareRobustCommand#1{%
    \ifmmode\def\next{\operatorname{#2}}%
      \else\def\next/{\textsc{\lowercase{#2}}}%
    \fi\next}}

\newmathcronym{\DAE}{DAE}
\newmathcronym{\INDCPA}{IND-CPA}
\newmathcronym{\PRF}{PRF}
\newmathcronym{\PRP}{PRP}
\newmathcronym{\SIV}{SIV}

\DeclareMathOperator{\Adv}{Adv}
\DeclareMathOperator{\DaenceDecrypt}{\textsc{Daence-Decrypt}}
\DeclareMathOperator{\DaenceEncrypt}{\textsc{Daence-Encrypt}}
\DeclareMathOperator{\muDAE}{mu-DAE}
\DeclareMathOperator{\muPRF}{mu-PRF}
\DeclareMathOperator{\trunc}{trunc}

\newcommand{\Z}{\mathbb Z}

\newcommand{\concat}{\mathbin\|}
\newcommand{\given}{\mathrel|}

\newcommand{\maxlength}{\ell}
\newcommand{\collisionbound}{\varepsilon}

\begin{document}

\maketitle

\begin{abstract}
  We present \Daence/, a deterministic authenticated cipher based on a
   pseudorandom function family and a universal hash family, similar
   to \SIV/~\cite{rogaway-shrimpton2006keywrap}.
  We recommend instances with
   \Salsa20/~\cite{bernstein2007salsa20} or
   \ChaCha/~\cite{bernstein2008chacha}, and
   \Poly1305/~\cite{bernstein2005poly1305},
   for high performance, high security, and easy deployment.

  \warning{The definition and analysis are works in progress.}
\end{abstract}

\section{Introduction}

The nonce-based authenticated cipher
 crypto\_secretbox\_xsalsa20poly1305 in
 \NaCl/~\cite{bernstein2009naclcrypto}, and the variant
 \ChaCha//\Poly1305/ defined by the \IETF/~\cite{rfc7539} for
 \TLS/~\cite{rfc7905}, are widely available in fast software
 implementations resistant to timing side channels.
The nonce-based authenticated cipher \AESGCM/~\cite{nist-sp800-38d} is
 popular, though only with hardware support is it fast and resistant
 to timing side channels.

These nonce-based ciphers fail catastrophically in the face of nonce
 reuse.
They are best suited to protocols that are designed to support
 sequential message numbers, such as the record number in \TLS/.
Some applications are unable to keep the state needed to maintain a
 sequential message number, and although they could use an extended
 nonce like \XSalsa20/~\cite{bernstein2011xsalsa} chosen randomly,
 some environments may not have a reliable entropy source.
The \emph{nonce-misuse-resistant} authenticated cipher
 \AESGCMSIV/~\cite{rfc8452} was developed to address these use cases,
 but it carries with it the performance and side channel costs of
 \AESGCM/---and amplifies the performance cost by deriving fresh keys
 for each distinct nonce, yet has very narrow security margins.

We propose a deterministic authenticated cipher \Daence/ built out of
 the \Salsa20/ or \ChaCha/ pseudorandom function family and the
 \Poly1305/ universal hash family.
The design is based on the \SIV/ construction of Rogaway and
 Shrimpton~\cite{rogaway-shrimpton2006keywrap}, with a
 variable-input-\PRF/ made by composing a universal hash family with a
 fixed-input \PRF/~%
  \cite[\S1.5]{bellare-canetti-krawczyk1996cascade}%
  \cite[\S9, Theorem~9.2]{bernstein1999hash127}.
\Daence/ is easily implemented in terms of the primitives available in
 \NaCl/ and \libsodium/.

\section{Security contract}

\Daence/ is a deterministic authenticated cipher.
This means it consists of two functions:
%
\begin{description}
  \item[\normalfont $t \concat c = \DaenceEncrypt(k, a, m)$] takes
     a key $k$ (96-byte in \Salsa20/-\Daence/, 64-byte in
      \ChaCha/-\Daence/),
     a string $a$ of at most $2^{38}$ bytes of associated data, and
     a message $m$ of at most $2^{38}$ bytes ($256\,\mathrm{GB}$).

    $\DaenceEncrypt$ returns an authenticated ciphertext $t \concat c$
     which is 24 bytes longer than $m$.

  \item[\normalfont $m = \DaenceDecrypt(k, a, t \concat c)$] takes
     a key $k$, a string $a$ of at most $2^{38}$ bytes of associated
     data, and an alleged authenticated ciphertext $t \concat c$ of
     at most $24 + 2^{38}$ bytes.

    \begin{itemize}
      \item If $t \concat c = \DaenceEncrypt(k, a, m)$,
         $\DaenceDecrypt$ returns $m$.
      \item Otherwise, $\DaenceDecrypt$ reports a forgery with high
         probability.
        In the sequel we denote this by the symbol `$\bot$'; in
         practice, a \texttt{crypto\_dae\_salsa20daence\_open}
         function in the style of \NaCl/ may return an error code or
         throw an exception.
    \end{itemize}
\end{description}

\paragraph*{Responsibilities of the user.}

\begin{enumerate}
  \item You must choose a secret key $k$ uniformly at random and
     independently of everything else in your application.
    (You may safely derive the 96-byte key $k$ from a 32-byte key $k'$
     by a key derivation function---\eg,
     $k = \operatorname{HKDF-SHA256}_{k'}(\mathit{salt}, \text{`foo'},
      96)$.)

  \item If $\DaenceDecrypt$ reports a forgery, you must decline to act
     on the alleged message content except by immediately dropping it
     on the floor.

  \item You must encrypt at most $2^{52}$ messages with each key.

    Applications limited to smaller \emph{messages} can safely process
     a larger \emph{number} of messages: if every message is at most
     $L < 2^{38}$ bytes long (including associated data), then you may
     process up to $2^{90}\!/L$ messages.
\end{enumerate}

\paragraph*{Security guarantee.}

Under any key $k$ independently:

\begin{enumerate}
  \item If you choose a distinct associated data string for every
     message, then an adversary has no hope of distinguishing the
     ciphertexts of your messages from uniform random byte strings of
     the same length, or of distingushing the tags from uniform random
     24-byte strings.

  \item If you repeat an associated data string, then an adversary has
     no hope of distinguishing the ciphertexts of \emph{distinct}
     messages from uniform random byte strings of the same length, but
     can tell when messages are repeated.

  \item
    \margintodo{Triple-check bounds!}
    The adversary's probability of succeeding at forgery---even
     after flooding your system with up to $2^{100}$ forgery
     attempts---is less than $1/2^{32}$.
\end{enumerate}

\subsection{Safe usage limits}

We recommend that each key be used to encrypt no more than $2^{52}$
 messages if an application may attain the maximum length of $2^{38}$
 bytes in the associated data and $2^{38}$ bytes in the message.
Such applications can withstand well over $2^{100}$ forgery attempts
 before the adversary's advantage exceeds $1/2^{32}$.

For better security, we recommend that each application set
 application-specific limits on the sum $\ell_a + \ell_m$ of the
 associated data and message length.
This limit should be chosen according to how much memory the
 application is willing to let a forger waste in a denial of service
 attack before detection.
For example:
\begin{itemize}
  \item an IP packet (on a path with the standard ethernet MTU)
     is at most $1500$ bytes;
  \item a TLS record is at most $2^{14}$ bytes;
  \item a file system block is typically at most $2^{15}$ or $2^{16}$ bytes;
  \item a Tarsnap file chunk can be up to around $2^{18}$ bytes.
\end{itemize}
Applications limited to smaller \emph{messages} can safely process a
 larger \emph{number} of messages per key---\eg, applications limited
 to $2^{30}$ bytes per message can safely process $2^{60}$ messages and
 withstand $2^{120}$ forgery attempts, and applications limited to 1500
 bytes per message can safely process $2^{79}$ messages and withstand
 $2^{158}$ forgery attempts, before the adversary's advantage exceeds
 $1/2^{32}$.
In general, a single \Daence/ key is safe for up to
 $2^{90}\!/(\ell_a + \ell_m)$
 messages, or $2^{80}$ messages, whichever is smaller.

\subsection{Comparison to alternative ciphers}

In \autoref{tab-security}, we compare the adversary's advantage against
 \Daence/ to several obvious alternatives, for various maximum message
 sizes and numbers of messages:

\begin{description}
  \item[\AESSIV/] (`\SIV/')
     a deterministic authenticated cipher in the \SIV/ construction,
     built out of \AESCTR/ and \AESCMAC/.
    (A variant, \AESPMACSIV/~\cite{arcieri2018aespmacsiv}, uses
     \AESPMAC/~\cite{black-rogaway2002pmac-eurocrypt} instead of
     \AESCMAC/ for better parallelism and essentially the same
     security.)

    Security
     (\cite[\S4, Theorem~2]{rogaway-shrimpton2006keywrap};
      \cite[\S5, Theorem~3]{rogaway-shrimpton2006keywrap},
       with $p=1$; and
      \cite[\S2, Theorem~2.2]{bernstein2005permutations})
     is dominated by the birthday bound on the 128-bit block size of
     \AES/.
    \Daence/ avoids this bottleneck by using a native \PRF/---\Salsa20/
     or \ChaCha/---instead of approximating one by a \PRP/ like \AES/.

  \item[\AESGCMSIV/] (`\GCMSIV/')
     a nonce-misuse-resistant authenticated cipher built out of
     \AESCTR/ and the polynomial evaluation hash \POLYVAL/ related to
     \AESGCM/'s \GHASH/.

    We consider both \AESGCMSIV/'s \emph{deterministic} security with a
     fixed nonce
     (\cite[\S4, Theorem~3]{iwata-seurin2017gcmsivsecurity}, with
      256-bit \AES/ keys and $Q = 1$ distinct nonces)
     as well as its \emph{randomized} security
     (\cite[\S3.3, Corollary~1]{iwata-seurin2017gcmsivsecurity},
      with 256-bit \AES/ keys and random nonces).

    Obviously \AESGCMSIV/'s randomized security, with no misuse, is
     better; we consider both because the cipher is advertised as
     `misuse-resistant'---if misused to the point that the same nonce
     is used for every message, whether because of entropy failure or
     virtual machine rollbacks or coding errors, \AESGCMSIV/ gives the
     deterministic security.
    The margin of this paper is too small to fit a neat visual
     representation of all the dimensions in which nonce reuse might
     occur---maximum reuses per nonce, average reuses per nonce, \etc.
    \begin{itemize}
      \item For \emph{deterministic security}, \AESGCMSIV/ is dominated
         by the birthday bound on the 128-bit block size of \AES/---and
         the birthday bound on the 128-bit tag size \emph{multiplied}
         by the maximum message length, because of \POLYVAL/ collision
         probability.

        \Daence/ avoids these bottlenecks by using a native \PRF/ and
         by using a much larger universal hash family to derive a
         192-bit tag.

      \item For \emph{randomized security with no misuse}, \AESGCMSIV/
         avoids the birthday bound on the block and tag size by
         deriving a fresh \AES/ key for each message---at high cost
         (whether or not it is misused) to performance and side channel
         security in implementations that cannot rely on hardware \AES/
         acceleration.
    \end{itemize}
    The deterministic security bound could be made tighter by
     judiciously incorporating the better \PRF//\PRP/ switching lemma
     of~\cite[\S2, Theorem~2.2]{bernstein2005permutations}
     in~\cite[Appendix~A, Lemma~2]{iwata-seurin2017gcmsivsecurity},
     but it would remain slightly worse than \AESSIV/ because
     \POLYVAL/'s collision probability contributes a factor of the
     maximum message length.

  \item[\AESGCM/] (`\GCM/')
     a nonce-based authenticated cipher built out of \AESCTR/ and the
     polynomial evaluation hash \GHASH/.

    Security
     (\cite[Appendix~C, Eq.~(22)]{iwata-ohashi-minematsu2012gcmsecurity})
     is dominated by the birthday bound on the 128-bit block size of
     \AES/.
    No misuse-resistance---repeating a nonce is fatal.

  \item[\ChaCha//\Poly1305/] (`\textsc{c/p}')
     a nonce-based authenticated cipher built out of \ChaCha/ and
     \Poly1305/.

    Security
     (\cite[\S9, Security notes]{bernstein2009naclcrypto})
     is \emph{not} affected by any birthday bound, but the \Poly1305/
     forgery advantage denominator, $2^{103}$, is smaller than for
     \GHASH/, $2^{128}$, which is why advantage against
     \ChaCha//\Poly1305/ starts higher but rises slower than \AESGCM/.
    No misuse-resistance---repeating a nonce is fatal.

    Unlike \ChaCha//\Poly1305/, \Daence/ \emph{is} subject to the
     birthday bound (on the \SIV/ tag size) because it has no
     nonce---but it drives the denominator up to $2^{206}$ by running
     two independent \Poly1305/ instances in parallel, which still
     leaves room to truncate the tag to 192 bits.
\end{description}

\begin{table}
  \caption{Comparison of adversary's advantage for various
     authenticated ciphers.}
  \label{tab-security}

  We suppose for the sake of presentation on two-dimensional paper that
   the adversary can attempt about a million times as many forgeries
   ($2^{20}$) as there are legitimate messages encrypted by the user;
   users for whom this is not accurate can use the theorems below to
   compute safe usage limits.

  Advantage bounds above $2^{-32}$ are
   \textcolor{Vermillion}{\textbf{\textit{highlighted}}};
   this somewhat arbitrary cutoff is derived from \NIST/ guidance on
   \AESGCM/~\cite[\S8, p.~18]{nist-sp800-38d}.

  \begin{center}
  \begin{minipage}{\textwidth}
  \renewcommand{\footnoterule}{}
  \newcommand{\bad}[1]{\textcolor{Vermillion}{\mathbfit{#1}}}
  \centerline{%
  \begin{tabular}{rccccccc}
    \multicolumn{2}{c}{}
    & \multicolumn{3}{c}{deterministic}
    & \multicolumn{1}{c}{randomized}
    & \multicolumn{2}{c}{nonce-based}
    \\
    max bytes per msg
    & msgs
    % deterministic
    & \Daence/
    & \SIV/
    & \GCMSIV/
    % randomized
    & \GCMSIV/
    % nonce-based
    & \GCM/
    & \textsc{c/p}
    \\
    \cmidrule(lr){1-2} \cmidrule(lr){3-5} \cmidrule(lr){6-6} \cmidrule(lr){7-8}
    \input{adv}
  \end{tabular}%
  }
  \end{minipage}
  \end{center}
\end{table}

\section{Definition}

\paragraph*{$\Poly1305/^2$.}
For 16-byte strings $k_1, \dotsc, k_4$, and a byte string $m$, define
\[
  \Poly1305/^2_{k_1,k_2}(m)
  := \Poly1305/_{k_1}(m) \concat \Poly1305/_{k_2}(m),
\]
 a 32-byte string.
For byte strings $a$ and $m$, define
\[
  \Poly1305/^2_{k_1,k_2,k_3,k_4}(a, m)
  := \Poly1305/^2_{k_3,k_4}(h_a \concat h_m),
\]
 where
 $h_a = \Poly1305/^2_{k_1,k_2}(a)$
 and
 $h_m = \Poly1305/^2_{k_1,k_2}(m)$.

\paragraph*{\HXSalsa20/.}
For 32-byte key $k_0$ and 16-byte inputs $i$ and $j$, define
\[
  \HXSalsa20/_{k_0}(i \concat j) := \HSalsa20/_{\HSalsa20/_{k_0}(i)}(j),
\]
 a 32-byte string.
Note there is a public function $G$ such that
 $\HXSalsa20/_{k_0}(i \concat j) = G(\XSalsa20/_{k_0}(i \concat j), j)$,
 just as \HSalsa20/ can be defined in terms of
 \Salsa20/~\cite{bernstein2011xsalsa} by
 $\HSalsa20/_{k_0}(i) = G(\Salsa20/_{k_0}(i), i)$.

\paragraph*{\Salsa20/-\Daence/.}
To \term{encrypt}, given
 a 96-byte key $k$,
 associated data $a$ of at most $2^{38}$ bytes, and
 a message $m$ of at most $2^{38}$ bytes,
 compute:
%
\begin{enumerate}
  \item $k_0 \concat k_1 \concat k_2 \concat k_3 \concat k_4 := k$
    \hfill (32-byte $k_0$; 16-byte $k_1, \dotsc, k_4$)
  \item $t := \trunc_{192}\bigl(
      \HXSalsa20/_{k_0}(\Poly1305/^2_{k_1,k_2,k_3,k_4}(a, m))
    \bigr)$
    \hfill (24-byte $t$)
  \item $c := m \oplus \bigl(\XSalsa20/_{k_0}(t \concat 0) \concat
      \XSalsa20/_{k_0}(t \concat 1) \concat \dotsb\bigr)$
\end{enumerate}
%
The authenticated ciphertext is $t \concat c$.

To \term{decrypt}, given
 a 96-byte key $k$,
 associated data $a$ of at most $2^{38}$ bytes, and
 an alleged authenticated ciphertext $t \concat c$ of $24$
 to $24 + 2^{38}$ bytes, compute:
%
\begin{enumerate}
  \item $k_0 \concat k_1 \concat k_2 \concat k_3 \concat k_4 := k$
    \hfill (32-byte $k_0$; 16-byte $k_1, \dotsc, k_4$)
  \item $m := c \oplus \bigl(\XSalsa20/_{k_0}(t \concat 0) \concat
      \XSalsa20/_{k_0}(t \concat 1) \concat \dotsb\bigr)$
  \item $t' := \trunc_{192}\bigl(
      \HXSalsa20/_{k_0}(\Poly1305/^2_{k_1,k_2,k_3,k_4}(a, m))
    \bigr)$
    \hfill (24-byte $t'$)
\end{enumerate}
%
If $t' \stackrel?= t$, return $m$; otherwise erase $m$ and report a
 forgery.

\paragraph*{\ChaCha/-\Daence/.}
\ChaCha/-\Daence/ has a 64-byte key, replaces \Salsa20/ by \ChaCha/
 (extended \XChaCha/~\cite{irtf-cfrg-xchacha-03}), and tweaks the
 encoding to match \ChaCha//\Poly1305/~\cite{rfc7539} by zero-padding
 the associated data and message to multiples 16 bytes as
 $\underline a$ and $\underline m$ and appending 8-byte little-endian
 lengths:
%
\begin{enumerate}
  \item $k_0 \concat k_1 \concat k_2 := k$
    \hfill (32-byte $k_0$, 16-byte $k_1$, 16-byte $k_2$)
  \item $t := \trunc_{192}\bigl(
      \HXChaCha/_{k_0}(
        \Poly1305/^2_{k_1,k_2}(
          \underline a \concat \underline m
            \concat \lvert a\rvert_8 \concat \lvert m\rvert_8))
    \bigr)$
  \item $c := m \oplus \bigl(\XChaCha/_{k_0}(t, 0) \concat
      \XChaCha/_{k_0}(t, 1) \concat \dotsb\bigr)$
\end{enumerate}
%
The \XChaCha/ input $(t, n)$ is encoded as
 $t_0 \concat n \concat 0^4 \concat t_1$,
 where
 $t_0$ is the first 16 bytes of $t$,
 $n$ is a 4-byte little-endian block counter,
 $0^4$ is four zero bytes, and
 $t_1$ is the last 8 bytes of $t$.
Decryption is derived similarly.

\section{Security notions}

Let $A$ be a random decision algorithm with access to an oracle
 $\mathcal O$.
Write $\Pr[A(\mathcal O)]$ for the probability $A$ accepts after
 making various queries to $\mathcal O$:
%
\begin{description}
  \item[Encryption queries for a \emph{single} user]
    Given associated data $a$ and message $m$, return an authenticated
     ciphertext $t \concat c$.
  \item[Decryption queries for a \emph{single} user]
    Given associated data $a$ and alleged authenticated ciphertext
     $t \concat c$, return a message $m$, or $\bot$ if the alleged
     authenticated ciphertext is deemed a forgery.
  \item[Encryption queries for \emph{multiple} users]
    Given a user number $u$, associated data $a$, and message $m$,
     return an authenticated ciphertext $t \concat c$.
  \item[Decryption queries for \emph{multiple} users]
    Given a user number $u$, associated data $a$, and alleged
     authenticated ciphertext $t \concat c$, return a message $m$, or
     $\bot$ if the alleged ciphertext is deemed a forgery.
  \item[Function queries for \emph{multiple} users]
    Given a user number $u$ and an input $x$, return an output $y$.
\end{description}
%
The adversary $A$ is assumed not to repeat queries, nor to submit the
 answers from encryption queries as decryption queries.
We review standard notions of security
 (\eg,~%
  \cite[Definition~1]{rogaway-shrimpton2006keywrap}%
  \cite[\S3]{bellare-tackmann2016mu-ae-crypto}),
 loosely summarized as how well $A$ can tell users of a real
 cryptosystem---\Daence/, \XSalsa20/, \etc---from pranksters who just
 roll dice to answer every query.

\begin{definition}
  The \term{pathological deterministic authenticated cipher $U_1$}
   returns an independent uniform random authenticated ciphertext of
   the appropriate length for each encryption query, and returns
   $\bot$ for every decryption query.
  The notion extends naturally to the multi-user setting; call it
   $U$.
\end{definition}

\begin{definition}
  For a deterministic authenticated cipher $E_k$ with random key $k$,
   the
   \term{multi-user deterministic authenticated encryption advantage}
   of $A$ against $E$ is the statistical distance from $E_k$ to $U$
   measured by $A$, where by abuse of language $k$ is understood to
   mean a collection of keys chosen independently by many users and
   $E_k$ is understood to mean a collection of instances of $E$ for
   many users keyed by their respective keys:
  \[
    \Adv^{\muDAE}_E(A) := \lvert\Pr[A(E_k)] - \Pr[A(U)]\rvert.
  \]
  Here $A$ may submit encryption and decryption queries for
   \emph{multiple} users.
\end{definition}

The single-user $\Adv^{\DAE}_E(A)$ is defined similarly.

\begin{definition}
  For a function family $\phi_k$ with random key $k$, the
   \term{multi-user pseudorandom function advantage}
   of $A$ against $\phi$ is the statistical distance from $\phi_k$ to
   $f$ measured by $A$, where $f$ is a a uniform random function of
   the same domain and codomain (where `$\phi_k$' and `$f$' again
   mean many independent instances):
  \[
    \Adv^{\muPRF}_\phi(A) := \lvert\Pr[A(\phi_k)] - \Pr[A(f)]\rvert.
  \]
  Here $A$ may submit function queries for \emph{multiple} users.
\end{definition}

\section{Analysis}

\begin{theorem}[\Salsa20/-\Daence/]\label{thm-salsa20-daence}
  Let $A$ be a random decision algorithm with encryption and
   decryption oracles for a set of deterministic authenticated
   cipher users.
  Suppose $A$ submits $E(u)$ encryption queries and $D(u)$ decryption
   queries to the $u^{\mathit{th}}$ user of up to $\ell_a(u)$ bytes of
   associated data and $\ell_m(u)$-byte messages.
  Then there is an algorithm $A'$ making at most
   $\sum_u \bigl(1 + \lceil\ell_m(u)/64\rceil\bigr)
     \bigl(E(u) + D(u)\bigr)$
   oracle queries and having the cost of $A$ plus the cost of
   evaluating $\Poly1305/^2$ and $\oplus$ on
   $\sum_u E(u) + D(u)$
   different $(\ell_a(u), \ell_m(u))$-byte inputs,
   such that
%
  \begin{multline*}
    \Adv^{\muDAE}_{\Daence/}(A)
     \leq \Adv^{\muPRF}_{\XSalsa20/}(A') \\
            + \sum_u
                \frac{2 D(u) + E(u)^2 + \binom{E(u)}{2}}
                     {2^{192}} \\
                + \collisionbound(\ell_a(u), \ell_m(u))
                  \cdot
                  {\textstyle\bigl(D(u) + \binom{E(u)}{2}\bigr)},
  \end{multline*}
%
   where
  \[
    \collisionbound(\ell_a, \ell_m)
    := \frac{\max\{\lceil\ell_a/16\rceil^2, \lceil\ell_m/16\rceil^2\}
             + 16}
            {2^{206}}.
  \]
\end{theorem}

\begin{theorem}[\ChaCha/-\Daence/]\label{thm-chacha-daence}
  Let $A$ be a random decision algorithm with encryption and
   decryption oracles for a set of deterministic authenticated
   cipher users.
  Suppose $A$ submits $E(u)$ encryption queries and $D(u)$ decryption
   queries to the $u^{\mathit{th}}$ user of up to $\ell_a(u)$ bytes of
   associated data and $\ell_m(u)$-byte messages.
  Then there is an algorithm $A'$ making at most
   $\sum_u \bigl(1 + \lceil\ell_m(u)/64\rceil\bigr)
     \bigl(E(u) + D(u)\bigr)$
   oracle queries and having the cost of $A$ plus the cost of
   evaluating $\Poly1305/^2$ and $\oplus$ on
   $\sum_u E(u) + D(u)$
   different $(\ell_a(u), \ell_m(u))$-byte inputs,
   such that
%
  \begin{multline*}
    \Adv^{\muDAE}_{\Daence/}(A)
     \leq \Adv^{\muPRF}_{\XChaCha/}(A') \\
            + \sum_u
                \frac{2 D(u) + E(u)^2 + \binom{E(u)}{2}}
                     {2^{192}} \\
                + \frac{\lceil(\ell_a(u) + \ell_m(u) + 16)/16\rceil^2}{2^{206}}
                  \cdot
                  {\textstyle\bigl(D(u) + \binom{E(u)}{2}\bigr)}.
  \end{multline*}
\end{theorem}

\paragraph*{Outline of proof.}

\begin{enumerate}
  \item Set a bound on the collision probability of $\Poly1305/^2$.
  \item Set a bound on \DAE/ advantage against an idealized version of
     \Daence/.
  \item Extend the bound to the multi-user setting.
  \item Instantiate the idealization with the actual \PRF/, \XSalsa20/
     or \XChaCha/.
\end{enumerate}

\subsection{Collisions under double-hashing with associated data}

\begin{lemma}[Double-hashing]\label{lem-double-hash}
  Let $k_1, k_2$ be independent \Poly1305/ keys.
  For any distinct strings $m \ne m'$ of at most $\ell$ bytes,
  \[
    \Pr[\Poly1305/^2_{k_1,k_2}(m) = \Poly1305/^2_{k_1,k_2}(m')]
    \leq \collisionbound(\ell)
    := \frac{\lceil\ell/16\rceil^2}{2^{206}}.
  \]
\end{lemma}

\begin{proof}
  By~\cite[Theorem~3.3]{bernstein2005poly1305},
  \[
    \Pr[\Poly1305/_{k_1}(m) = \Poly1305/_{k_1}(m')]
    \leq \frac{8\lceil\ell/16\rceil}{2^{106}},
  \]
   and likewise for $k_2$.
  Since $k_1$ and $k_2$ are independent,
  \begin{align*}
    \Pr&[\Poly1305/^2_{k_1,k_2}(m) = \Poly1305/^2_{k_1,k_2}(m')] \\
    &= \Pr[\Poly1305/_{k_1}(m) = \Poly1305/_{k_1}(m'), \\
    &\qquad\quad\Poly1305/_{k_2}(m) = \Poly1305/_{k_2}(m')] \\
    &= \Pr[\Poly1305/_{k_1}(m) = \Poly1305/_{k_1}(m')] \\
    &\quad
       \cdot
       \Pr[\Poly1305/_{k_2}(m) = \Poly1305/_{k_2}(m')] \\
    &\leq \biggl(\frac{8\lceil\ell/16\rceil}{2^{106}}\biggr)^2
     = \frac{64\lceil\ell/16\rceil^2}{2^{212}}
     = \frac{\lceil\ell/16\rceil^2}{2^{206}}.
    \qedhere
  \end{align*}
\end{proof}

\begin{lemma}[Hashing tuples]\label{lem-hash-tuple}
  Let $k_1, \dotsc, k_4$ be independent \Poly1305/ keys.
  For strings $a, a'$ up to $\ell_a$ bytes and $m, m'$ up to $\ell_m$
   bytes, if $(a, m) \ne (a', m')$ then
%
  \begin{multline*}
    \Pr[\Poly1305/^2_{k_1,k_2,k_3,k_4}(a, m)
        = \Poly1305/^2_{k_1,k_2,k_3,k_4}(a', m')] \\
    \leq \collisionbound(\ell_a, \ell_m)
    := \frac{\max\{\lceil\ell_a/16\rceil^2, \lceil\ell_m/16\rceil^2\}
             + 16}
            {2^{206}}.
  \end{multline*}
\end{lemma}

\begin{proof}
  Write
   $H := \Poly1305/^2_{k_1,k_2}$ and
   $H^* := \Poly1305/^2_{k_3,k_4}$.
  Let $h_x := H(x)$.
  We must have either $a \ne a'$ or $m \ne m'$, or both.
  If $a \ne a'$, then by \autoref{lem-double-hash},
   $\Pr[(h_a, h_m) = (h_{a'}, h_{m'})]
    \leq \Pr[h_a = h_{a'}]
    \leq \collisionbound(\ell_a)$.
  Similarly, if $m \ne m'$, the probability is bounded by
   $\collisionbound(\ell_m)$,
   so in either case,
  \[
    \Pr[(h_a, h_m) = (h_{a'}, h_{m'})]
    \leq \max\{\collisionbound(\ell_a), \collisionbound(\ell_m)\}.
  \]
  Finally, since $H^*$ is independent of $H$ and thus of the $h_x$:
%
  \begin{align*}
    \Pr&[H^*(h_a \concat h_m) = H^*(h_{a'} \concat h_{m'})] \\
    &\leq \Pr[(h_a, h_m) = (h_{a'}, h_{m'})] \\
    &\qquad
          + \Pr[H^*(h_a \concat h_m) = H^*(h_{a'} \concat h_{m'})
                \given (h_a, h_m) \ne (h_{a'}, h_{m'})] \\
    &\leq \max\{\collisionbound(\ell_a), \collisionbound(\ell_m)\}
          + \collisionbound(64)
     = \frac{\max\{\lceil\ell_a/16\rceil^2, \lceil\ell_m/16\rceil^2\}
             + 4^2}
            {2^{206}}.
    \qedhere
  \end{align*}
\end{proof}

\newpage

\subsection{Idealizing the cipher}

Let $f\colon \{0,1\}^{256} \to \{0,1\}^{512}$ be a uniform random
 function, an idealization of $\XSalsa20/$ under a uniform random key.
Denote by
 $\hat f\colon \{0,1\}^{256} \to \{0,1\}^{256}$
 the function
 $\hat f(n_0 \concat n_1) := G(f(n_0 \concat n_1), n_1)$,
 for 16-byte $n_0$ and $n_1$,
 \ie, the corresponding idealization of $\HXSalsa20/$ under the
 \emph{same} key.
For 192-bit $t$, denote by $f_*(t)$ the concatenation
 $f(t \concat 0) \concat f(t \concat 1) \concat f(t \concat 2) \concat \dotsb$,
 with the 64-bit input counter $0, 1, 2, \dotsc$
 (limited below $2^{32}$)
 encoded in little-endian---this idealizes the \XSalsa20/ stream
 cipher, again under the \emph{same} key.
Note that $f$, $\hat f$, and $f_*$ are all uniformly distributed
 (but not independent of one another!).

Define \term{\Deuce/} to be as \Daence/, but with $f$ substituted for
 $\XSalsa20/_{k_0}$:
%
\begin{enumerate}
  \item $h := \Poly1305/^2_{k_1,k_2,k_3,k_4}(a, m)$
    \hfill (32-byte $h$)
  \item $t := \trunc_{192}(\hat f(h))$
    \hfill (24-byte $t$)
  \item $c := m \oplus f_*(t)$
  \item return $t \concat c$
\end{enumerate}
%
(Decryption is defined the obvious way.)

\begin{lemma}[Single-user idealized security]\label{lem-su-ideal}
  Let $A$ be a random decision algorithm with encryption and
   decryption oracles for a deterministic authenticated cipher.
  Suppose $A$ makes $E$ encryption queries and $D$ decryption queries
   of up to $\ell_a$ bytes of associated data and $\ell_m$-byte
   messages.
  Then
  \[
    \Adv^{\DAE}_{\Deuce/}(A)
    \leq \frac{2 D + E^2 + \binom{E}{2}}{2^{192}}
         + \collisionbound(\ell_a, \ell_m)
           \cdot
           {\textstyle\bigl(D + \binom{E}{2}\bigr)}.
  \]
\end{lemma}

\begin{proof}

Recall
 $\Adv^{\DAE}_{\Deuce/}(A) := \lvert\Pr[A(\Deuce/)] - \Pr[A(U_1)]$
 where $U_1$ is the pathological `DAE' whose encryption oracle always
 answers with independent uniform random bit strings and whose
 decryption oracle always answers with $\bot$.
We will show that queries to a \Deuce/ oracle likewise all return
 independent uniform random bit strings for encryption and $\bot$ for
 decryption, except in events of low probability.

Setting aside forgery attempts (decryption queries) for the moment,
 consider the following set of encryption queries to a \Deuce/ oracle
 and its responses:
\[
  (a_1, m_1) \mapsto t_1 \concat c_1, \quad
  (a_2, m_2) \mapsto t_2 \concat c_2, \quad
  \cdots, \quad
  (a_E, m_E) \mapsto t_E \concat c_E,
\]
 where
\[
\begin{aligned}
  h_i &= \Poly1305/^2_{k_1,k_2,k_3,k_4}(a_i, m_i), \\
  t_i &= \trunc_{192}(\hat f(h_i)) = \trunc_{192}(G(f(h_i), h_i)), \\
  c_i &= m_i \oplus f_*(t_i)
       = m_i \oplus \bigl(
           f(t_i \concat 0) \concat
           f(t_i \concat 1) \concat
         \dotsb\bigr).
\end{aligned}
\]
Note that for each $i$, both $t_i$ and $c_i$ are uniform random on
 their own.
Further, as long as all the \emph{inputs} to $f$, via $\hat f$ and
 $f_*$, are \emph{distinct}, all the \emph{outputs}---and thus all the
 $t_i \concat c_i$---are \emph{independent}.
Conditional on this event, the distribution on the encryption oracle
 query responses for \Deuce/ is exactly the same as the distribution on
 encryption oracle query responses for $U_1$---independent uniform
 random bit strings.
Conditional also on the event that there are no successful forgeries,
 \Deuce/ and $U_1$ have exactly the same distribution for \emph{all}
 queries.

Among the oracle queries, either a successful forgery must happen
 first, or a collision in the inputs to $f$ must happen first.
It will be convenient to study the probability of a collision between a
 hash value $h_i$ and a tag/counter input $t_j \concat n$, separately
 from the probability of tag collision $t_i = t_j$ for some $i \ne j$.
So we will break the `bad' events into:
%
\begin{enumerate}
  \item
    $C_{th} := \exists i, j, n.\; h_i = t_j \concat n$
    ---
     a collision between a hash $h_i$ and a tag/counter
     $t_j \concat n$,
     \emph{before} a successful forgery and
     \emph{before} a collision among the tags.
  \item
    $C_t := \exists i \ne j.\; t_i = t_j$
    ---
     a collision between two tag values $t_i = t_j$ for $i \ne j$,
     \emph{before} a hash and tag/counter collision and
     \emph{before} a successful forgery.
  \item
    $F$
    ---
     a successful forgery
     \emph{before} a collision in the inputs to $f$.
\end{enumerate}
%
If none of these happen---no collision in $f$ inputs, no
 forgery---\Deuce/ and $U_1$ have the same distribution, so
%
\begin{align*}
  \Pr[A(\Deuce/)]
  &\leq \Pr[A(\Deuce/) \given \lnot C_{th}, \lnot C_t, \lnot F]
     + \Pr[C_{th}] + \Pr[C_t] + \Pr[F] \\
  &= \Pr[A(U_1)] + \Pr[C_{th}] + \Pr[C_t] + \Pr[F],
\end{align*}
%
 and since $A$ was arbitrary and could be replaced by $\lnot A$, we
 have
\[
  \lvert\Pr[A(\Deuce/)] - \Pr[A(U_1)]\rvert
  \leq \Pr[C_{th}] + \Pr[C_t] + \Pr[F].
\]
We will next show that the events $C_{th}$, $C_t$, and $F$ occur only
 with low probability.

For upper bounds on $\Pr[C_{th}]$ and $\Pr[C_t]$, we can assume $A$
 makes no forgery attempts---if it attempted forgeries
 (decryption queries)
 interspersed with encryption queries, we could construct an adversary
 $A'$ that is identical except without the forgery attempts.
The probabilities $\Pr[C_{th}]$ and $\Pr[C_t]$ of collisions
 \emph{before} a successful forgery are at least as high for $A'$,
 which never even attempts a forgery and so can never be interrupted in
 its quest for collisions by a successful forgery, as for $A$, which
 may succeed at forgery before it has had the opportunity to submit all
 the encryption queries that $A'$ would have submitted.
(This follows the argument
 of~\cite[Theorem~1]{dodis-pietrzak2007randmac}.)

\paragraph*{Probability of hash and tag/counter collision.}
To set an \emph{upper} bound on the probability of
 $h_i = t_j \concat n$ for some $i,j,n$,
 it suffices to consider the event that the first 192 bits
 $\trunc_{192}(h_i)$
 of hash $h_i$ coincides with tag $t_j$
 (before a successful forgery),
 ignoring the counter $n$:
%
\begin{multline*}
  \Pr[C_{th}]
   = \Pr[\exists i,j,n.\; h_i = t_j \concat n]
   \leq \Pr[\exists i,j.\; \trunc_{192}(h_i) = t_j] \\
   \leq \sum_{i,j} \Pr[\trunc_{192}(h_i) = t_j]
   = \sum_{i,j} \frac{1}{2^{192}}
   = \frac{E^2}{2^{192}},
\end{multline*}
%
 where
 $\Pr[\trunc_{192}(h_i) = t_j] = 1/2^{192}$
 because $t_j = f(h_j)$ is a uniform random 192-bit string
 independent of $\trunc_{192}(h_i)$.
(This holds even if $i = j$, because $f$ is a uniform random
 function---its output on any input is independent of the input.)

\paragraph*{Probability of tag collision.}
The collision $t_i = t_j$ for $i \ne j$ in the event $C_t$ may arise
 either from the hashes
 $h_i = \Poly1305/^2_{k_1,k_2,k_3,k_4}(a_i, m_i)$,
 or from the tags
 $t_i = f(h_i)$
 in encryption queries.
Let $C_h$ be the event of a collision $h_i = h_j$ for $i \ne j$
 (before a successful forgery, and before a hash and tag/counter
 collision).
Since $\Poly1305/^2$ has collision probability bounded by
 $\collisionbound(\ell_a, \ell_m)$,
 we have
%
\begin{align*}
  \Pr[C_h]
    &= \Pr[\exists i<j\colon h_i = h_j]
     \leq \sum_{i<j} \Pr[h_i = h_j] \\
    &= \sum_{i<j}
         \Pr[\Poly1305/^2_{k_1,k_2,k_3,k_4}(a_i, m_i)
               = \Poly1305/^2_{k_1,k_2,k_3,k_4}(a_j, m_j)] \\
    &\leq \sum_{i<j} \collisionbound(\ell_a, \ell_m)
     = \binom{E}{2} \collisionbound(\ell_a, \ell_m).
\end{align*}
%
If the $h_i$ are all distinct, then the $t_i$ are uniform random bit
 strings independent of everything else involved---which means an
 adversary learns no new information from each query to adaptively act
 on in subsequent queries---so there is a collision among the $t_i$
 only with probability
%
\begin{multline*}
  \Pr[C_t \given \lnot C_h]
     = \Pr[\exists i<j\colon t_i = t_j \given \lnot C_h]
     \leq \sum_{i<j} \Pr[t_i = t_j \given \lnot C_h] \\
     = \sum_{i<j} \Pr[f(h_i) = f(h_j) \given h_i \ne h_j]
     \leq \sum_{i<j} \frac{1}{2^{192}}
     = \binom{E}{2} \frac{1}{2^{192}}.
\end{multline*}
%
Thus,
\[
  \Pr[C_t] \leq \Pr[C_h] + \Pr[C_t \given \lnot C_h]
    \leq \binom{E}{2} \biggl(
        \collisionbound(\ell_a, \ell_m)
        + \frac{1}{2^{192}}
      \biggr).
\]

\paragraph*{Forgery probability.}
To bound $\Pr[F]$, assume $A$ halts after the first forgery
 attempt---an adversary that makes $D>1$ forgery attempts can be broken
 into one that halts after the first forgery attempt, and another one
 that simulates forgery failure for the first attempt before making
 $D-1$ more attempts.
If the first succeeded with probability at most $\Pr[F_0]$, then by
 induction the second succeeds with probability at most
 $(D - 1) \cdot \Pr[F_0]$,
 so that the original succeeds with probability at most
 $\Pr[F_0] + (D - 1) \cdot \Pr[F_0] = D \cdot \Pr[F_0]$.

Let $(a', t' \concat c')$ be the single forgery attempt of $A$,
 with
 $t' \concat c' \ne t_i \concat c_i$
 for all $i$.
Let
 $m' = c' \oplus f_*(t')$
 and
 $h' = \Poly1305/^2_{k_1,k_2,k_3,k_4}(a', m')$.
The forgery succeeds if $t' = f(h')$.
Note that $(a', m') \ne (a_i, m_i)$, for if
 $(a', m') = (a_i, m_i)$,
 then we would have
 $t' \concat c' = t_i \concat c_i$, since---for fixed
 $f,k_1,\dotsc,k_4$ values---$t' \concat c'$ is a deterministic
 function of $(a', m')$.

Let $F_0^{\ne}$ be the event that the forgery is successful \emph{and}
 $t' \ne t_i$ for all $i$, and let $F_0^=$ be the event that the
 forgery is successful \emph{and} $t' = t_i$ for some $i$, so that
 $\Pr[F_0] = \Pr[F_0^{\ne}] + \Pr[F_0^=]$.
%
\begin{itemize}
  \item
    If $t' \ne t_i$ for all $i$, we must have $h' \ne h_i$ for all $i$,
     since $t' = f(h')$ and $t_i = f(h_i)$.
    Hence $f(h')$ is a uniform random 192-bit string independent of all
     the $t_i$ and $c_i$, so $t' = f(h')$ only with probability
     $1/2^{192}$ for this forgery attempt; that is,
    \[
      \Pr[F_0^{\ne}] = 1/2^{192}.
    \]
  \item
    If $t' = t_i$ for some $i$, then necessarily $m' \ne m_i$ for all
     $i$, and the forgery succeeds only when $h' = h_i$, which occurs
     with probability at most $\collisionbound(\ell_a, \ell_m)$; or, if
     $h' \ne h_i$, when $f(h') = f(h_i)$, which occurs with probability
     $1/2^{192}$.
    Thus,
    \[
      \Pr[F_0^=] \leq \collisionbound(\ell_a, \ell_m) + 1/2^{192}.
    \]
\end{itemize}
%
Combining these, we see the probability of a forgery (before any
 collisions) is bounded by
\[
  \Pr[F]
  \leq D \cdot \Pr[F_0]
  \leq D \cdot \Pr[F_0^{\ne}] + D \cdot \Pr[F_0^=]
  \leq \frac{2 D}{2^{192}} + D \cdot \collisionbound(\ell_a, \ell_m).
\]

\paragraph*{Summing it up.}

\begin{align*}
  \lvert\Pr&[A(\Deuce/)] - \Pr[A(U_1)]\rvert
   \leq \Pr[C_{th}] + \Pr[C_t] + \Pr[F] \\
  &\leq \frac{E^2}{2^{192}}
     + \binom{E}{2} \biggl(
         \collisionbound(\ell_a, \ell_m)
         + \frac{1}{2^{192}}
       \biggr)
     + \frac{2 D}{2^{192}} + D \cdot \collisionbound(\ell_a, \ell_m) \\
  &\leq \frac{2 D + E^2 + \binom{E}{2}}{2^{192}}
     + \collisionbound(\ell_a, \ell_m)
       \cdot
       {\textstyle\bigl(D + \binom{E}{2}\bigr)}.
  \qedhere
\end{align*}

\end{proof}

\subsection{Multi-user security}

\begin{lemma}[Multi-user idealized security]\label{lem-mu-ideal}
  Let $A$ be a random decision algorithm with encryption and
   decryption oracles for a \emph{set} of deterministic authenticated
   cipher users.
  Suppose $A$ makes $E(u)$ encryption queries and $D(u)$ decryption
   queries to the $u^{\mathit{th}}$ user of up to $\ell_a(u)$ bytes of
   associated data and $\ell_m(u)$-byte messages.
  Then
%
  \begin{multline*}
    \Adv^{\muDAE}_{\Deuce/}(A)
    \leq \sum_u \frac{2 D(u) + E(u)^2 + \binom{E(u)}{2}}
                     {2^{192}} \\
           + \collisionbound(\ell_a(u), \ell_m(u))
             \cdot
             {\textstyle\bigl(D(u) + \binom{E(u)}{2}\bigr)}.
  \end{multline*}
\end{lemma}

\begin{proof}
  In the foregoing analysis of the idealized cipher \Deuce/ in the
   single-user setting (\autoref{lem-su-ideal}), the probabilities of
   the critical events---$C_{th}$, $C_t$, and $F$---can be
   straightforwardly seen to sum over the users.
  For example, let $C_t^u$ be the event of a tag collision for the
   $u^{\mathit{th}}$ user, and let $C_t^*$ be the event of a tag
   collision in \emph{any one} of the users; then
  \[\textstyle
    \Pr[C_t^*] = \Pr[\exists u.\; C_t^u] \leq \sum_u \Pr[C_t^u].
  \]
  Consequently, the multi-user \DAE/ advantage against \Deuce/---that
   is, the statistical distance under $A$ from \Deuce/ to a
   pathological `\DAE/' collection $U$---is at most the sum of the
   single-user \DAE/ advantages.
\end{proof}

\paragraph*{Discussion.}
The generic single-user-to-multi-user security lemma%
 ~\cite[Lemma~3.3]{bellare-canetti-krawczyk1996cascade}
  \cite[Lemma~1]{bellare-bernstein-tessaro2016amac-mu}
 shows that if an adversary making a total of $q$ queries to \emph{one}
 user has advantage at most $\epsilon$, then an adversary making a
 total of $q$ queries distributed across $N$ users has advantage at
 most $N\cdot\epsilon$.
Here, with $q = \sum_u E(u) + D(u)$, we have shown a multi-user
 advantage bound much smaller than the generic lemma yields as
 a corollary.
Loosely, where the generic would be
 $N\cdot \bigl(\sum_u \cdots\bigr)^2\!/2^\lambda$,
 we show
 $\sum_u (\cdots)^2\!/2^\lambda$
 instead.

In \Deuce/, a query to a never-before-queried user gives a response
 independent of all prior query responses, and so queries to new users
 contribute only \emph{linearly} to the adversary's advantage.
But a query to a previously-queried user may lead to a collision
 involving that user's secret $f$ or $\Poly1305/^2_{k_1,\dotsc,k_4}$,
 and so by the birthday bound repeated queries to the same user
 contribute \emph{quadratically} to the adversary's advantage.
Thus, the best strategy at breaking the idealized cipher---recall this
 is before we have instantiated it with \XSalsa20/ so there is no batch
 key search advantage yet---is to focus on the user with the highest
 maximum message length or bandwidth, ignoring all other users.

\subsection{Instantiating the idealized cipher}

\begin{lemma}[Multi-user instantiation]\label{lem-mu-instance}
  Let $A$ be a random decision algorithm with encryption and
   decryption oracles for a \emph{set} of deterministic authenticated
   cipher users.
  Suppose $A$ makes $E(u)$ encryption queries and $D(u)$ decryption
   queries to the $u^{\mathit{th}}$ user of up to $\ell_a(u)$ bytes of
   associated data and $\ell_m(u)$-byte messages.
  Then there is an algorithm $A'$ making at most
   $\sum_u \bigl(1 + \lceil\ell_m(u)/64\rceil\bigr)
     \bigl(E(u) + D(u)\bigr)$
   oracle queries and having the cost of $A$ plus the cost of
   evaluating $\Poly1305/^2$ and $\oplus$ on
   $\sum_u E(u) + D(u)$
   different $(\ell_a(u), \ell_m(u))$-byte inputs,
   such that
  \[
    \lvert\Pr[A(\Daence/)] - \Pr[A(\Deuce/)]\rvert
    = \Adv^{\muPRF}_{\XSalsa20/}(A').
  \]
\end{lemma}

\begin{proof}
  \Deuce/ is simply \Daence/ with a uniform random function $f$
   substituted for $\XSalsa20/_{k_0}$, so if $A$ can distinguish a
   collection of \Daence/ users from a collection of \Deuce/ users then
   it can be used in an algorithm $A'$ to distinguish a collection of
   $\XSalsa20/_{k_0}$ users from a collection of $f$ users (recall
   $f\colon \{0,1\}^{256} \to \{0,1\}^{512}$ is a
   uniform random function).

  If $\mathcal O(u,x)$ is an oracle for a collection of function
   instances indexed by $u$, define $A'(\mathcal O)$ to run $A$ with
   oracles for the following \DAE/ under independent uniform random keys
   $k_1,\dotsc,k_4$ for the $u^{\mathit{th}}$ user, decryption being
   defined the obvious way:
%
  \begin{enumerate}
    \item $h := \Poly1305/^2_{k_1,k_2,k_3,k_4}(a, m)$
      \hfill (32-byte $h$)
    \item $t := \trunc_{192}(\hat{\mathcal O}(u, h))$
      \hfill (24-byte $t$)
    \item $c := m \oplus \mathcal O_*(u, t)$
    \item return $t \concat c$
  \end{enumerate}
%
  This invokes the oracle once for each message
   (via $\hat{\mathcal O}$)
   and once for each 64-byte block in each message
   (via $\mathcal O_*$).
  Then $A'(\XSalsa20/_{k_0}) = A(\Daence/)$ and $A'(f) = A(\Deuce/)$
   (with `$k_0$' and `$f$' understood to mean a collection of
   independent keys/functions), so
  %
  \begin{align*}
    \lvert\Pr[A(\Daence/)] - \Pr[A(\Deuce/)]\rvert
    &= \lvert\Pr[A'(\XSalsa20/_{k_0})] - \Pr[A'(f)]\rvert \\
    &= \Adv^{\muPRF}_{\XSalsa20/}(A').
    \qedhere
  \end{align*}
\end{proof}

\subsection{Tying the room together}

\begin{proof}[Proof of \autoref{thm-salsa20-daence}]
  By stringing all the inequalities together, we complete the proof:
%
  \begin{align*}
    \Adv^{\muDAE}_{\Daence/}(A)
    &= \lvert\Pr[A(\Daence/)] - \Pr[A(U)]\rvert \\
    &\leq \lvert\Pr[A(\Daence/)] - \Pr[A(\Deuce/)]\rvert \\
    &\qquad + \lvert\Pr[A(\Deuce/)] - \Pr[A(U)]\rvert \\
    &\leq \Adv^{\muPRF}_{\XSalsa20/}(A') \\
    &\qquad
       + \sum_u \frac{2 D(u) + E(u)^2 + \binom{E(u)}{2}}
                     {2^{192}} \\
    &\qquad\qquad\qquad
           + \collisionbound(\ell_a(u), \ell_m(u))
             \cdot
             {\textstyle\bigl(D(u) + \binom{E(u)}{2}\bigr)}.
    \qedhere
  \end{align*}
\end{proof}

\begin{proof}[Proof of \autoref{thm-chacha-daence}]
  The \ChaCha/ version of \Daence/ uses
  \[
    \Poly1305/^2_{k_1,k_2}(
      a \concat m \concat \lvert a\rvert_8 \concat \lvert m\rvert_8)
    \qquad\text{instead of}\qquad
    \Poly1305/^2_{k_1,k_2,k_3,k_4}(a, m).
  \]
  By \autoref{lem-double-hash}, the collision probability is bounded
   by $\collisionbound(\ell_a + \ell_m + 16)$ rather than by
   $\collisionbound(\ell_a, \ell_m)$; the rest of the analysis carries
   over identically, with \XChaCha/ in the place of \XSalsa20/.
\end{proof}

\newpage
\section{Shaving IAQs---Infrequently Asked Questions}

\begin{description}
  \item[Why not just use the \CAESAR/ competition winner?]
    In our estimation, the \CAESAR/
     competition\footnote{\CAESAR/: Competition for Authenticated
        Encryption: Security, Applicability, and Robustness.
       \url{https://competitions.cr.yp.to/caesar.html}}
     was too broad and dragged on for too long, and in the end failed
     to gain the traction it needed.
    While \NaCl/, \IETF/ \ChaCha//\Poly1305/, and \AESGCM/ are
     ubiquitous today, with a variety of high-quality implementations
     available in many programming languages, the benefits of the
     \CAESAR/ winners do not seem to justify the engineering effort to
     make the novel cryptographic primitives as ubiquitous.
    In contrast, \Daence/ requires negligible engineering effort on
     top of \NaCl/ or \libsodium/.

  \item[Why not just use \AESGCMSIV/?]
    \AESGCMSIV/ is optimized for applications that can
     \emph{guarantee} hardware support for the primitives---otherwise
     it may be subject to severe performance degradation and/or timing
     side channel attacks.
    This may be reasonable when the engineer can control everything
     about the hardware and software stack, and audit the software
     stack all the way down to the hardware to ensure safety, but it's
     less appealing for a general-purpose tool.

    The \AESGCMSIV/ security guarantee requires unusually detailed
     safe usage limits~\cite[\S9]{rfc8452}, and users are advised to
     choose nonces at random---which is exactly the opposite of the
     advice for nonce-based ciphers like \AESGCM/ and \NaCl/
     \naclsecretbox/.
    This limits \AESGCMSIV/'s value as a drop-in replacement for
     nonce-based ciphers with a safety net for \Tarsnap/-style
     accidental nonce reuse
     bugs~\cite{cperciva2011tarsnap-nonce-reuse} and virtual machine
     rollbacks, and makes applications \emph{more} vulnerable to
     broken entropy sources than \Daence/ would.

  \item[Why not a prior \ChaCha//\Poly1305/-based design?]
    There have been past designs for nonce-misuse-resistant
     ciphers built with \ChaCha/ and \Poly1305/, such as HS1-SIV in
     \CAESAR/~\cite{krovetz2015hs1siv}
     and \XChaCha20/-HMAC-SHA256-SIV proposed as a now-withdrawn
     internet-draft~\cite{madden-generalised-siv-00}.

    HS1-SIV relies on nonstandard security assumptions about \ChaCha/,
     requires a large key of 176 bytes at the low end or 368 bytes
     for security comparable to \Daence/, and by using polynomial
     evaluation modulo $2^{61} - 1$ can't take advantage of existing
     high-speed constant-time \Poly1305/ logic---it is not just a few
     lines of code to call out to \ChaCha/ and \Poly1305/.

    \XChaCha20/-HMAC-SHA256-SIV is a more conservative design, but it
     uses the expensive and overpowered HMAC-SHA256 rather than taking
     advantage of cheap \Poly1305/ code that is likely to be hanging
     out in the neighbourhood of \ChaCha/ code.

    Further, despite being flavours of \SIV/, both HS1-SIV and
     \XChaCha20/-HMAC-SHA256-SIV still require nonces, which we feel is
     an unnecessary complication, addressed below.

  \item[Why not a nonce-misuse-resistant authenticated cipher?]
    If you have a nonce---say a message sequence number---then you can
     use it in the associated data to conceal repeated messages like a
     nonce-based cipher does
    So out of \Daence/ you can build your own nonce-misuse-resistant
     authenticated cipher: just prefix a fixed-size nonce to the
     associated data!
    You can also safely incorporate a \emph{secret}
     nonce~\cite{bellare-ng-tackmann2019nonces-revisited-crypto} into
     the encrypted payload.

    But there is a conceptual cost to \emph{requiring} a nonce
     parameter \emph{and} associated data---what are you supposed to
     put in the nonce and what are you supposed to put in the
     associated data?
    The terms `nonce' and `nonce-misuse-resistant authenticated
     cipher' are confusing to non-experts (not to mention the
     British).

    The security bounds could be better if we derived a fresh key for
     each distinct nonce like \AESGCMSIV/ does.
    But \AESGCMSIV/ has weaker security bounds to begin with.
    At the level of security that \Daence/ provides, there is little
     reason to pay the interface complexity cost.

  \item[Why bother with \Salsa20/?  Why not just \ChaCha/?]
    \NaCl/ and many of its derivatives support only \Salsa20/, not
     \ChaCha/, and for \Poly1305/ only have a one-shot
     \naclonetimeauth/ function that processes a whole message at
     once.
    It is easy to implement \Salsa20/-\Daence/ in terms of this
     interface, but not \ChaCha/-\Daence/.

    On the other hand, \ChaCha//\Poly1305/ is seeing wider use after
     \IETF/ standardization in \TLS/---and systems with
     \ChaCha//\Poly1305/ will generally have the parts needed to
     implement \ChaCha/-\Daence/ just as easily.

    There is a small security advantage to the hashing in
     \Salsa20/-\Daence/; there may be a small performance advantage to
     the hashing in \ChaCha/-\Daence/; but the primary advantage to
     defining both alternatives is to reduce the engineering costs to
     adopting either one in environments that already use \NaCl/ or
     already implement \ChaCha//\Poly1305/.

  \item[Why not \ChaCha12/?]
    The reduced-round variant \ChaCha12/ is faster than \ChaCha20/
     (\ie, \ChaCha/ with the default number of rounds, 20), and still
     has a comfortable security margin from the best attacks on
     reduced-round \ChaCha/ in the literature to date.
    This is why, for instance, Google selected \ChaCha12/ in Adiantum
     disk encryption~\cite{crowley-biggers2018adiantum}.

    However, \ChaCha20/ is much more widely available in
     libraries today---\eg,
     \libsodium/,
     \OpenSSL/,
     \BearSSL/,
     \Nettle/,
     the \Rust/ \texttt{crypto::chacha20} module,
     the \Go/ \texttt{golang.org/x/crypto/chacha20} package,
     and
     the \Python/ pyca cryptography.io library,
     none of which provide \ChaCha12/.

    Similarly, \Salsa{20/12}/ was selected over \Salsa{20/20}/ in the
     \eSTREAM/ portfolio for the same performance improvement with a
     comfortable security margin, but also seems to be much less widely
     deployed than \Salsa{20/20}/.

    So although in principle \ChaCha12/ or \Salsa{20/12}/ would improve
     performance over \ChaCha20/ or \Salsa{20/20}/ at negligible
     security cost, they appear to raise the engineering costs of
     deployment in practice.

  \item[Why not use \Auth256/?]
    The \Auth256/~\cite{bernstein-chou2014auth256} message
     authentication code is based on a universal hash family with
     collision probability bounded by $1/2^{255}$ using a key as long
     as the message.
    This bound obviously seems better than the
     ${\approx}\ell^2\!/2^{206}$ bound for $\Poly1305/^2$, so why not
     reach for it?

    \margintodo{Double-check bounds!}
    With \Daence/, even if an application allowed messages up to the
     maximum length, $2^{39}$ ($2^{38}$ bytes of associated data and
     $2^{38}$ bytes of message), exposed quadrillions of legitimate
     messages ($2^{52}$) to the adversary, and the adversary attempted
     an unimaginable $2^{100}$ forgeries, the probability of \emph{one}
     forgery would stay below $1/2^{32}$.
    So there is little \emph{security} motivation to replace
     \Poly1305/ by a larger hash.

    What about performance?
    \Auth256/, even at its best on very long messages, is not faster
     than \Poly1305/ in software.
    Maybe \Auth256/ would improve on $\Poly1305/^2$---it's not clear,
     \latin{a priori}, and since \Auth256/ requires a message-length
     key the cost would have to figure in key generation.

    But most importantly, neither \Auth256/ nor any other
     ${\approx}256$-bit universal hash family, whether in a binary
     field or large prime field, is widely implemented and deployed
     the way \Poly1305/ is.
    So---even if there may be a slim performance
     improvement---switching from \Poly1305/ would substantially raise
     the engineering costs of adopting \Daence/.

  \item[How fast is it?]
    The main cost over \naclsecretbox/ is evaluating \Poly1305/ twice
     rather than once, and completing it before starting \XSalsa20/,
     so \Salsa20/-\Daence/ should cost about 1--2x what
     \naclsecretbox/ costs.
    If the analysis survives some scrutiny, we will submit \Daence/ to
     \SUPERCOP/\footnote{\url{https://bench.cr.yp.to/}} for reliable,
     fair measurements across a variety of machines.

    But if you must see rough numbers first, on our Intel
     Kaby~Lake~i7, \SUPERCOP/ measures
     ${\approx}2.5\,\mathrm{cpb}$ for \naclsecretbox/,
     ${\approx}1.9\,\mathrm{cpb}$ for hardware-accelerated \AESGCM/,
     and
     ${\approx}3.6\,\mathrm{cpb}$ for \Salsa20/-\Daence/.
    There is room for improvement: this naïve code makes no attempt to
     compute the two \Poly1305/'s in parallel, which may be faster
     than computing them serially---both for messages exceeding the
     \CPU/ cache, and for short messages for which setting up many
     powers of the evaluation point is not worthwhile.

  \item[Why a 96-byte key (or 64-byte for \ChaCha/-\Daence/)?]
    For \ChaCha/-\Daence/, in addition to \ChaCha/ we use two
     independent \Poly1305/ instances---each requiring 16-byte keys,
     for a total of $32 + 2\cdot16 = 64$ bytes of key material---in
     order to provide high security without nonces.

    For \Salsa20/-\Daence/, in addition to \Salsa20/ we use
     \emph{four} independent \Poly1305/ instances---each requiring
     16-byte keys, for a total of $32 + 4\cdot16 = 96$ bytes of key
     material---primarily to support incorporating associated data
     without relying an incremental-update API for \Poly1305/.
    This way, \Salsa20/-\Daence/ in order to take advantage of the
     existing deployed \NaCl/ library which can only compute
     \Poly1305/ on an entire message.

    We could start from a 32-byte master key and then derive subkeys
     from it.
    However, doing this with the obvious tool at hand---either
     \Salsa20/ or \ChaCha/---would add hundreds of cycles to the cost
     of processing each message.
    Since many applications already use trees of key derivation for
     various purposes, we feel that the cost of deriving another few
     dozen bytes of key material---which can be cached and reused by
     the application---is not worth the cost of hundreds of additional
     cycles per message.

  \item[Does \Daence/ support streaming?]
    If you want to stream large files, break them into bite-size
     pieces to be encrypted as separate messages.
    In the associated data, include: a unique file name, so the pieces
     can't be swapped between different files; a piece number, so the
     pieces of a file can't be reordered; and an end-of-stream flag, so
     you can detect truncation.

    You should ensure the pieces are no larger than the amount of
     memory you are willing to let an adversary waste with a forgery in
     a denial of service attack.
    Smaller message size limits also give better \Daence/ security.
    Some typical message size limits:
    \begin{itemize}
      \item an IP packet is at most $1500$ bytes;
      \item a TLS record is at most $2^{14}$ bytes;
      \item a file system block is typically at most $2^{15}$ or $2^{16}$
        bytes;
      \item a Tarsnap file chunk can be up to around $2^{18}$ bytes.
    \end{itemize}

  \item[How is the `ae' in \Daence/ pronounced?]
    Like the `a' in `data'.
\end{description}

\noindent
We do not anticipate that \Daence/ will replace \ChaCha//\Poly1305/ in
 major protocols such as \TLS/ designed by world-class cryptographers,
 which can easily take advantage of a message number guaranteed not to
 repeat.
We do hope that \Daence/ will find its way into the repertoire of
 general-purpose application engineers who need to store messages safe
 from eavesdropping and forgery in diverse software
 environments---%
 \emph{without} auditing the software stack all the way down to
  machine instructions to ensure their use of \AESGCMSIV/ is safe from
  timing side channels,
 \emph{without} adopting unusual cryptographic primitives or an
  entirely new cryptography library, and
 \emph{without} teetering on the brink of catastrophe from nonce reuse
  in \naclsecretbox/ or \AESGCM/.

\bibliography{daence}

\newpage
\appendix

\section{Tweetable implementation}

\lstinputlisting[basicstyle=\small\ttfamily,tabsize=4]{tweetdaence.c}

\section{Reference implementation}

\lstinputlisting[basicstyle=\small\ttfamily,tabsize=4,formfeed=\newpage]
  {kat_salsa20daence.c}

\newpage
\section{Test vectors}

\lstinputlisting[basicstyle=\small\ttfamily]{kat_salsa20daence.out}

\end{document}

%%  LocalWords:  pseudorandom crypto secretbox xsalsa Rogaway dae SHA
%%  LocalWords:  Shrimpton ciphertext daence HKDF ciphertexts decrypt
%%  LocalWords:  endian cryptosystem codomain th IAQs cryptographic
%%  LocalWords:  nonces quintillion priori Kaby cpb subkeys Tweetable
%%  LocalWords:  naïve ae basicstyle tabsize tweetdaence formfeed kat
