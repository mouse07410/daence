\documentclass{article}

\usepackage{amsfonts}
\usepackage{amsmath}
\usepackage{amsthm}
\usepackage[T1]{fontenc}
\usepackage[final=true,backref=page,colorlinks=true]{hyperref}[2016/05/21]
\usepackage[final]{listings}

\title{\DAENCE/:
  \Salsa20/ and \ChaCha/ in Deterministic Authenticated Encryption with a noNCE}
\author{Taylor `Riastradh' Campbell}

\bibliographystyle{plain}
\renewcommand*{\backrefalt}[4]{%
  \ifcase#1% no citations
  \or% one citation
    Cited on p.~#2.%
  \else% multiple citations
    Cited on pp.~#2.%
  \fi%
}

\newtheorem{definition}{Definition}
\newtheorem{lemma}{Lemma}
\newtheorem{theorem}{Theorem}
\newcommand{\term}[1]{\textbf{#1}}

\newcommand{\margintodo}[1]{%
  \leavevmode\marginpar{\leavevmode%
    \parskip=4pt plus 2pt minus 1pt\relax%
    \color{red}%
    TODO: #1%
  }%
}

\def\operatorsc#1{{%
  \ifmmode\let\next=\operatorname\else\let\next=\relax\fi\next{\textsc{#1}}}}
\def\Salsa#1/{\operatorsc{Salsa#1}}
\def\ChaCha#1/{\operatorsc{ChaCha#1}}
\def\HChaCha#1/{\operatorsc{HChaCha#1}}
\def\HSalsa#1/{\operatorsc{HSalsa#1}}
\def\XChaCha#1/{\operatorsc{XChaCha#1}}
\def\HXChaCha#1/{\operatorsc{HXChaCha#1}}
\def\XSalsa#1/{\operatorsc{XSalsa#1}}
\def\HXSalsa#1/{\operatorsc{HXSalsa#1}}
\def\Poly#1/{\operatorsc{Poly#1}}
\def\Auth#1/{\operatorsc{Auth#1}}
\def\DAENCE/{\operatorsc{Daence}}
\def\DUNCE/{\operatorsc{Dunce}}
\def\DEUCE/{\operatorsc{Deuce}}
\def\DOOZE/{\operatorsc{Dooze}}
\def\DICE/{\operatorsc{Dice}}

\def\CAESAR/{\textsc{Caesar}}
\def\OCB/{OCB}
\def\NaCl/{NaCl}
\def\IETF/{IETF}
\def\TLS/{TLS}
\def\IV/{\textsc{iv}}
\def\AESGCM/{AES\nobreakdash-GCM}
\def\AESGCMSIV/{AES\nobreakdash-GCM\nobreakdash-SIV}
\def\naclsecretbox/{crypto\_secretbox\_xsalsa20poly1305}
\def\naclonetimeauth/{crypto\_onetimeauth\_poly1305}
\def\libsodium/{libsodium}
\def\sodiumaead/{crypto\_aead}

\makeatletter
\def\newlatin#1#2{\def#1{\protect\@latin{#2}}}
\def\@latin#1{\@ifnextchar.{\@latinfinal{#1}}{\@latinmedial{#1}}}
\def\@latinfinal#1{\emph{#1}\@}
\def\@latinmedial#1{\emph{#1.}}
\makeatother
\newlatin\eg{e.g}
\newlatin\etc{etc}
\newlatin\etseq{et seq}
\newlatin\ie{i.e}
\newlatin\vs{vs}

\def\INDCPA{%
  \ifmmode\def\next{\operatorname{IND-CPA}}%
    \else\def\next/{\textsc{ind-cpa}}%
  \fi\expandafter\next}

\def\PRF{%
  \ifmmode\def\next{\operatorname{PRF}}%
    \else\def\next/{\textsc{prf}}%
  \fi\next}

\def\muPRF{%
  \ifmmode\def\next{\operatorname{mu-PRF}}%
    \else\def\next/{mu-\textsc{prf}}%
  \fi\next}

\def\PRG{%
  \ifmmode\def\next{\operatorname{PRG}}%
    \else\def\next/{\textsc{prg}}%
  \fi\next}

\def\SIV{%
  \ifmmode\def\next{\operatorname{SIV}}%
    \else\def\next/{\textsc{siv}}%
  \fi\next}

\def\DAE{%
  \ifmmode\def\next{\operatorname{DAE}}%
    \else\def\next/{\textsc{dae}}%
  \fi\next}

\def\AEAD{%
  \ifmmode\def\next{\operatorname{AEAD}}%
    \else\def\next/{\textsc{aead}}%
  \fi\next}

\def\NMRAE{%
  \ifmmode\def\next{\operatorname{NMRAE}}%
    \else\def\next/{\textsc{nmrae}}%
  \fi\next}

\def\muNMRAE{%
  \ifmmode\def\next{\operatorname{mu-NMRAE}}%
    \else\def\next/{mu-\textsc{nmrae}}%
  \fi\next}

\def\muDAE{%
  \ifmmode\def\next{\operatorname{mu-DAE}}%
    \else\def\next/{mu-\textsc{dae}}%
  \fi\next}

\DeclareMathOperator{\Adv}{Adv}
\DeclareMathOperator{\DaenceEncrypt}{\textsc{Daence-Encrypt}}
\DeclareMathOperator{\DaenceDecrypt}{\textsc{Daence-Decrypt}}

\newcommand{\Z}{\mathbb Z}

\newcommand{\concat}{\mathbin\|}
\newcommand{\given}{\mathrel|}

\newcommand{\maxlength}{\ell}
\newcommand{\collisionbound}{\varepsilon}

\begin{document}

\maketitle

\begin{abstract}
  We present \DAENCE/, a nonce-misuse-resistant authenticated cipher
   construction based on a 128-bit-to-512-bit pseudorandom function
   family and a universal hash family with low collision probability.
  We recommend instantiation using \Salsa20/ or \ChaCha/, and
   \Poly1305/.

  \textbf{\textcolor{red}{\textsc{Warning}: The definition and
       analysis are works in progress.}}
\end{abstract}

\section{Introduction}

The authenticated cipher crypto\_secretbox\_xsalsa20poly1305
 in \NaCl/~\cite{bernstein2009naclcrypto}, and the variant
 \ChaCha//\Poly1305/ defined by the \IETF/~\cite{rfc7539} for
 \TLS/~\cite{rfc7905}, are widely available in fast software
 implementations resistant to timing side channels.
The authenticated cipher \AESGCM/~\cite{nist-sp800-38d} is popular,
 though fast and resistant to timing side channels only with hardware
 support.

\naclsecretbox/ and \AESGCM/ fail catastrophically in the face of
 nonce reuse.
They are best suited to protocols that are designed to support
 sequential message numbers, such as the record number in \TLS/.
Some applications are unable to keep the state needed to maintain a
 sequential message number, and although they could use an extended
 nonce like \XSalsa20/ chosen randomly, some environments may not
 have a reliable entropy source.

The \emph{nonce-misuse-resistant} authenticated cipher
 \AESGCMSIV/~\cite{rfc8452} was developed to address these use cases,
 but it carries with it the performance and side channel costs of
 \AESGCM/---and amplifies the performance cost by deriving fresh keys
 for each distinct nonce, yet has very narrow security margins.

We propose a nonce-misuse-resistant authenticated cipher \DAENCE/,
 built out of a deterministic authenticated cipher using \Salsa20/ or
 \ChaCha/ and the \Poly1305/ universal hash family, with keys derived
 from a master key and a 128-bit nonce.
The design is based on the \SIV/ construction of Rogaway and
 Shrimpton~\cite{rogaway-shrimpton2006keywrap}, with a
 variable-input-PRF made by composing a universal hash family with a
 fixed-input PRF~%
  \cite[\S1.5]{bellare-canetti-krawczyk1996cascade}%
  \cite[\S9, Theorem.~9.1]{bernstein1999hash127}.
\DAENCE/ is easily implemented in terms of the primitives available in
 \NaCl/ and \libsodium/.

\section{Security contract}

\DAENCE/ is a nonce-misuse-resistant authenticated cipher.
This means it consists of two functions:
%
\begin{description}
  \item[\normalfont $c = \DaenceEncrypt(k, n, m)$] takes a 32-byte key
     $k$, a 16-byte nonce $n$, and a message $m$.

    $\DaenceEncrypt$ returns an authenticated ciphertext $c$ which is
     24 bytes longer than $m$.

  \item[\normalfont $m' = \DaenceDecrypt(k, n, c')$] takes a 32-byte
     key $k$, a 16-byte nonce $n$, and an alleged authenticated
     ciphertext $c'$.

    \begin{itemize}
      \item If $c' = \DaenceEncrypt(k, n, m)$, $\DaenceDecrypt$
         returns $m$.
      \item If $c'$ is a forgery, $\DaenceDecrypt$ returns $\bot$ with
         high probability.
    \end{itemize}

    In practice, a \texttt{crypto\_daencebox\_salsa20poly1305\_open}
     function in the style of \NaCl/ may return an error code or
     throw an exception to represent $\bot$.
\end{description}

\paragraph*{Responsibilities of the user.}

\begin{enumerate}
  \item You must choose a secret key $k$ uniformly at random.
  \item If $\DaenceDecrypt$ reports a forgery, you must decline to act
     on the alleged message content except by immediately dropping it
     on the floor.
  \item You must process no more than $2^{80}$ bytes of data with a
     single key.
\end{enumerate}

\paragraph*{Security guarantee.}

\begin{enumerate}
  \item Authenticated encryption, if you use a distinct nonce $n$ for
     every message.

    XXX Provide detailed safe usage limits here.

  \item Deterministic authenticated encryption, for all messages under
     any repeated nonce $n$.

    XXX Provide detailed safe usage limits here.
\end{enumerate}

\break

\section{Definition}

\paragraph*{Double-hashing.}
For 16-byte $k_1$ and 16-byte $k_2$, define
\[
  \Poly1305/^2_{k_1,k_2}(m)
  := \Poly1305/_{k_1}(m) \concat \Poly1305/_{k_2}(m).
\]
Note the collision probability of $\Poly1305/^2$ under independent
 keys is the square of the collision probability of \Poly1305/.

\paragraph*{\HXSalsa20/.}
For 32-byte key $k_0$ and 16-byte inputs $i$ and $j$, define
\[
  \HXSalsa20/_{k_0}(i \concat j) := \HSalsa20/_{\HSalsa20/_{k_0}(i)}(j).
\]
Note that \HXSalsa20/ can be defined in terms of \XSalsa20/ by
 choosing the appropriate words of the output and subtracting $j$,
 just as \HSalsa20/ can be defined in terms of \Salsa20/.

\paragraph*{Encryption.}
Given
 a 32-byte key $k$,
 a 16-byte message number $n$, and
 a message $m$ of fewer than $2^{70}$ bytes,
 compute:
%
\begin{enumerate}
  \item $k_0 \concat k_1 \concat k_2 := \Salsa20/_k(n)$
    \hfill (32-byte $k_0$, 16-byte $k_1$, 16-byte $k_2$)
  \item $t \concat \_ := \HXSalsa20/_{k_0}(\Poly1305/^2_{k_1,k_2}(m))$
    \hfill (24-byte $t$)
  \item $c := m \oplus \bigl(\XSalsa20/_{k_0}(t \mathbin\| 0) \concat
      \XSalsa20/_{k_0}(t \mathbin\| 1) \concat \dotsb\bigr)$
\end{enumerate}
%
The authenticated ciphertext is $t \concat c$.

\paragraph*{Decryption.}
Given
 a 32-byte key $k$,
 a 16-byte message number $n$, and
 an alleged authenticated ciphertext $t' \concat c'$ of fewer than
 $24 + 2^{70}$ bytes, compute:
%
\begin{enumerate}
  \item $k_0 \concat k_1 \concat k_2 := \Salsa20/_k(n)$
    \hfill (32-byte $k_0$, 16-byte $k_1$, 16-byte $k_2$)
  \item $m' := c' \oplus \bigl(\XSalsa20/_{k_0}(t \mathbin\| 0) \concat
      \XSalsa20/_{k_0}(t \mathbin\| 1) \concat \dotsb\bigr)$
  \item $t' \concat \_ := \HXSalsa20/_{k_0}(\Poly1305/^2_{k_1,k_2}(m'))$
    \hfill (24-byte $t'$)
\end{enumerate}
%
If $t' \stackrel?= t$, return $m'$; otherwise erase $m'$ and report a
 forgery.

\paragraph*{Associated data.}
\DAENCE/ may be extended to support associated data by uniquely
 encoding it in the input to \Poly1305/.
We recommend using \ChaCha/ instead of \Salsa20/ in order to take
 advantage of the components of existing implementations of the
 \ChaCha//\Poly1305/ authenticated cipher used in \TLS/ and
 implemented in \libsodium/ \sodiumaead/.
With associated data $a$ of $\lvert a\rvert < 2^{64}$ bytes,
 encryption is as above but with length-delimited associated data fed
 into $\Poly1305/^2$:
%
\begin{itemize}
  \item $k_0 \concat k_1 \concat k_2 := \ChaCha/_k(n)$
    \hfill (32-byte $k_0$, 16-byte $k_1$, 16-byte $k_2$)
  \item $t \concat \_ :=
      \HXChaCha/_{k_0}(
        \Poly1305/^2_{k_1,k_2}(\lvert a\rvert \concat a \concat m))$
    \hfill (24-byte $t$)
  \item $c := m \oplus \bigl(\XChaCha20/_{k_0}(t \mathbin\| 0) \concat
      \XChaCha20/_{k_0}(t \mathbin\| 1) \concat \dotsb\bigr)$
\end{itemize}
%
The corresponding decryption operation is derived similarly.

\section{Analysis}

\begin{theorem}
  Let $A_*$ be a random decision algorithm with access to encryption
   and decryption oracles for $U$ independent nonce-based
   authenticated ciphers with the nonce chosen by the adversary.
  Suppose the adversary reuses the $u^{\mathit{th}}$ user's nonce $n$
   in $E(u,n)$ encryption queries and $D(u,n)$ decryption queries
   (that is, forgery attempts).
  Suppose the queries are limited to messages up to $\ell$ bytes
   long.
  Then there exist
%
  \begin{itemize}
    \item a multi-user \PRF/ distinguisher $A'_*$ making
       $\sum_{u,n} E(u,n) + D(u,n)$
       oracle queries and having the cost of $A_*$ plus the cost of
       evaluating $\HXSalsa20/ \mathbin\circ \Poly1305/^2$,
       $\XSalsa20/$, and $\oplus$ on
       $\sum_{u,n} E(u,n) + D(u,n)$
       different $\ell$-byte inputs; and
    \item a multi-user \PRF/ distinguisher $A'$ making
       $\bigl(1 + \lceil\ell/64\rceil\bigr)
         \sum_{u,n} \bigl(E(u,n) + D(u,n)\bigr)$
       oracle queries and having the cost of $A_*$ plus the cost of
       evaluating $\Poly1305/^2$ and $\oplus$ on
       $\sum_{u,n} E(u,n) + D(u,n)$
       different $\ell$-byte inputs,
  \end{itemize}
%
   such that
%
  \begin{align*}
    \Adv^{\muNMRAE}_{\DAENCE/}(A_*)
    &\leq \Adv^{\muPRF}_{\Salsa20/}(A'_*)
          + \Adv^{\muPRF}_{\XSalsa20/}(A') \\
    &\qquad + \sum_{u,n}
                \frac{\bigl(E(u,n) + D(u,n)\bigr)^2
                       + D(u,n) + \binom{E(u,n)}{2}}
                     {2^{192}} \\
    &\qquad\qquad\qquad
            + 64\lceil\ell/16\rceil^2
              \frac{D(u,n) E(u,n) + \binom{E(u,n)}{2}}{2^{212}}.
  \end{align*}
\end{theorem}

\noindent
Outline of proof:

\begin{enumerate}
  \item Prove bounds on advantage against an idealization of the
     deterministic authenticated cipher, with a uniform random
     function in the place of \XSalsa20/.
  \item Extend the bound on advantage against the idealized \DAE/ to
     the multi-instance setting, where each instance has an
     independent uniform random function.
  \item Instantiate the idealization with \XSalsa20/, which adds only
     the multi-instance advantage against \XSalsa20/.
  \item Show that the multi-\emph{user} setting is equivalent to the
     multi-\emph{nonce} setting under idealized key derivation.
  \item Instantiate the idealized key derivation with \Salsa20/, which
     adds only the multi-instance advantage against \Salsa20/.
\end{enumerate}

\subsection{Idealizing the cipher}

Let $A$ be a random decision algorithm with access to encryption and
 decryption oracles for a deterministic authenticated cipher.
Assume $A$ makes $E$ queries to the encryption oracle and $D$ queries
 to the decryption oracle.
Assume messages are up to $\ell$ bytes long.

For a function $\phi$ yielding bit strings, denote by $\phi_{192}$ the
 first 192 bits of $\phi$; denote by $\hat\phi$ the transformation so
 that if $\phi = \Salsa20/$ then $\hat\phi = \HSalsa20/$; and denote
 by $\phi_*(t)$ the concatenation
 $\phi(t \concat 0) \concat \phi(t \concat 1) \concat \dotsb$,
 with input counter encoded in little-endian, and with output
 understood to be as long as is needed.
Let $f,g\colon \{0,1\}^{256} \to \{0,1\}^{512}$ be uniform random.
Note that $\hat f$, $\hat f_{192}$, $f_*$, and $g_*$ are uniformly
 distributed.

Define \term{$\DAENCE/_1$} to be as \DAENCE/, but keyed by
 $(k_0,k_1,k_2)$ directly, with no key derivation or nonce input.

Define \term{\DEUCE/} to be as $\DAENCE/_1$, but with $f$ substituted
 for $\XSalsa20/_{k_0}$:
%
\begin{enumerate}
  \item $h := \Poly1305/^2_{k_1,k_2}(m)$
    \hfill (32-byte $h$)
  \item $t := \hat f_{192}(h)$
    \hfill (24-byte $t$)
  \item $c := m \oplus f_*(t)$
\end{enumerate}

Define \term{\DICE/} to be as \DEUCE/, but with $g$ substituted for
 $f$ in the pad:
%
\begin{enumerate}
  \item $h := \Poly1305/^2_{k_1,k_2}(m)$
    \hfill (32-byte $h$)
  \item $t := \hat f_{192}(h)$
    \hfill (24-byte $t$)
  \item $c := m \oplus g_*(t)$
\end{enumerate}

Finally, define \term{\DUNCE/} to be as \DICE/, but with the
 decryption oracle modified to return $\bot$ for every input.

\paragraph*{Distinguishing \DEUCE/ from \DICE/: independent tags and pads.}
\leavevmode\hfill\break
\DICE/ is simply \DEUCE/ with two independent functions for deriving a
 tag $t$ from a hash $h$ for deriving a pad from a tag.
Only in the event $C_{th}$ of a collision between one of the values of
 $h$ and one of the values of $t \concat j$ do \DICE/ and \DEUCE/ have
 different distributions, since if the $h$ do not overlap with the
 $t \concat j$, then the $f(t \concat j)$ and $g(t \concat j)$ are
 both uniform random strings independent of the $\hat f_{192}(h)$.
And since $f(t \concat j)$ does not figure into \DICE/, a collision
 between a value of $h$ and a value of $t \concat j$ value cannot
 affect the outcome, so
 $\Pr[A(\DICE/) \given \lnot C_{th}] = \Pr[A(\DICE/)]$.
Thus,
%
\begin{align*}
  \Pr[A(\DEUCE/)]
  &= \Pr[A(\DEUCE/) \given C_{th}] \Pr[C_{th}] \\
  &\qquad
     + \Pr[A(\DEUCE/) \given \lnot C_{th}] \Pr[\lnot C_{th}] \\
  &\leq \Pr[C_{th}] + \Pr[A(\DEUCE/) \given \lnot C_{th}] \\
  &= \Pr[C_{th}] + \Pr[A(\DICE/) \given \lnot C_{th}] \\
  &= \Pr[C_{th}] + \Pr[A(\DICE/)].
\end{align*}

How often do collisions occur?
There are at most $E + D$ distinct values of $h$.
The values of $t$ are independent uniform random 192-bit strings, and
 there are up to $E + D$ of them, so
\[
  \Pr[C_{th}]
  \leq \sum_{i=1}^{E + D} \frac{E + D}{2^{192}}
  = \frac{(E + D)^2}{2^{192}}.
\]

Finally, since $A$ was arbitrary and could be replaced by $\lnot A$,
 we have
\[
  \lvert\Pr[A(\DEUCE/)] - \Pr[A(\DOOZE/)]\rvert
  \leq \Pr[C_{th}]
  \leq \frac{(E + D)^2}{2^{192}}.
\]

\paragraph*{Distinguishing \DICE/ from \DUNCE/: bounding forgery
  probability.}
\leavevmode\hfill\break
The decryption oracle of \DICE/ returns anything other than
 $\bot$ only in the event $F$ that $A$ forges a ciphertext not
 obtained from the encryption oracle.
Conditional on $\lnot F$, \DICE/ and \DUNCE/ otherwise have the same
 distribution---the encryption oracle is the same and the decryption
 oracle always returns $\bot$ in both cases---so
%
\begin{align*}
  \Pr[A(\DICE/)]
  &= \Pr[A(\DICE/) \given F] \Pr[F]
     + \Pr[A(\DICE/) \given \lnot F] \Pr[\lnot F] \\
  &\leq \Pr[F] + \Pr[A(\DICE/) \given \lnot F] \\
  &= \Pr[F] + \Pr[A(\DUNCE/)],
\end{align*}
%
 and thus
\[
  \lvert\Pr[A(\DICE/)] - \Pr[A(\DUNCE/)]\rvert
  \leq \Pr[F].
\]

Assume $A$ halts after the first forgery attempt---an adversary that
 makes $D>1$ forgery attempts can be broken into an adversary that
 halts after the first forgery attempt, and an adversary that
 simulates forgery failure for the first attempt before making $D-1$
 more attempts.
If the first variant succeeded with probability at most $p$, then by
 induction the second succeeds with probability at most $(D - 1) p$,
 so that the original adversary succeeds with probability at most
 $p + (D - 1) p = D p$.

Suppose $A$ submits distinct encryption queries
 $m_1, \dotsc, m_E$
 yielding answers
 $t_1 \concat c_1, \dotsc, t_E \concat c_E$
 via intermediate hashes
 $h_i = \Poly1305/^2_{k_1,k_2}(m_i)$,
 so that $t_i = f(h_i)$ and $c_i = m_i \oplus g_*(t_i)$.
Suppose $A$ then attempts a forgery $(t', c')$; let
 $m' = c' \oplus g_*(t')$
 and
 $h' = \Poly1305/^2_{k_1,k_2}(m')$.
The forgery succeeds if $m' \ne m_i$ for all $i$, and $t' = f(h')$.
Note that the string $f(h')$ is a uniform random 192-bit string
 independent of the $t_i$ and $c_i$, except in the event $C_{h'}$ that
 $h'$ coincides with one of the $h_i$.

Let
 $\collisionbound = (8\lceil\ell/16\rceil/2^{106})^2
   = 64\lceil\ell/16\rceil^2\!/2^{212}$
 be a bound on the collision probability of $\Poly1305/^2$.
Then $h'$ coincides with one of the $h_i$ with probability
%
\begin{align*}
  \Pr[C_{h'}]
  &= \Pr[\exists i\colon h' = h_i]
   \leq \sum_i \Pr[h' = h_i] \\
  &= \sum_i \Pr[\Poly1305/^2_{k_1,k_2}(m') = \Poly1305/^2_{k_1,k_2}(m_i)] \\
  &\leq \sum_i \collisionbound
   = E \collisionbound.
\end{align*}
%
Hence
%
\begin{align*}
  \lvert\Pr[A(\DICE/)] &- \Pr[A(\DUNCE/)]\rvert
   \leq \Pr[F] \\
  &\leq D \cdot \Pr\bigl[t' = f(h')
            \given t_i = f(h_i), c_i = m_i \oplus g_*(t_i)\bigr] \\
  &\leq D \cdot \bigl(\Pr[t' = f(h') \given \lnot C_{h'}]
                      + \Pr[C_{h'}]\bigr)
   \leq \frac{D}{2^{192}} + D E \collisionbound.
\end{align*}

\paragraph*{Distinguishing \DUNCE/ from uniform random.}
Let $U_1$ be a pathological `\DAE/' whose encryption oracle returns an
 independent uniform random tag and ciphertext for each distinct
 input, and whose decryption oracle returns $\bot$ for every input.
Suppose $A$ submits distinct encryption queries
 $m_1, \dotsc, m_E$
 yielding answers
 $t_1 \concat c_1, \dotsc, t_E \concat c_E$.
If the oracle is \DUNCE/, then all the strings $t_i = f(h_i)$ and
 $c_i = m_i \oplus g_*(t_i)$ are independent uniform random---and thus
 the oracle is indistinguishable from $U_1$---except in the event
 $C_h$ of a collision $h_i = h_j$ for some $i \ne j$, or in the event
 $C_t$ of a collision $t_i = t_j$ for some $i \ne j$.

First, since $\Poly1305/^2$ has collision probability bounded by
 $\collisionbound$,
%
\begin{align*}
  \Pr[C_h]
    &= \Pr[\exists i<j\colon h_i = h_j]
     \leq \sum_{i<j} \Pr[h_i = h_j] \\
    &= \sum_{i<j}
         \Pr[\Poly1305/^2_{k_1,k_2}(m_i) = \Poly1305/^2_{k_1,k_2}(m_j)] \\
    &\leq \sum_{i<j} \collisionbound
     = \binom{E}{2} \collisionbound.
\end{align*}
%
Next, assuming the $h_i$ are all distinct, there is a collision among
 the $t_i$ only with probability
%
\begin{align*}
  \Pr[C_t \given \lnot C_h]
    &= \Pr[\exists i<j\colon t_i = t_j \given \lnot C_h]
     \leq \sum_{i<j} \Pr[t_i = t_j \given \lnot C_h] \\
    &= \sum_{i<j} \Pr[f(h_i) = f(h_j) \given h_i \ne h_j] \\
    &\leq \sum_{i<j} \frac{1}{2^{192}}
     = \binom{E}{2} \frac{1}{2^{192}}.
\end{align*}
%
Thus,
%
\begin{align*}
  \Pr[A(\DUNCE/)]
  &\leq \Pr\bigl[A(\DUNCE/) \given
            \lnot (C_h \mathrel{\text{or}} C_t)\bigr]
        + \Pr[C_h \mathrel{\text{or}} C_t] \\
  &= \Pr[A(U_1)] + \Pr[C_h \mathrel{\text{or}} C_t] \\
  &\leq \Pr[A(U_1)] + \Pr[C_h] + \Pr[C_t \given \lnot C_h] \\
  &\leq \Pr[A(U_1)] + \binom{E}{2} \collisionbound
        + \binom{E}{2} \frac{1}{2^{192}},
\end{align*}
%
 so that
\[
  \lvert\Pr[A(\DUNCE/)] - \Pr[A(U_1)]\rvert
  \leq \binom{E}{2} \biggl(\collisionbound + \frac{1}{2^{192}}\biggr).
\]

\paragraph*{Summing it up.}

\begin{align*}
  \Adv^{\DAE}_{\DEUCE/}(A)
  &= \lvert\Pr[A(\DEUCE/)] - \Pr[A(U_1)]\rvert \\
  &\leq \lvert\Pr[A(\DEUCE/)] - \Pr[A(\DICE/)]\rvert \\
  &\qquad + \lvert\Pr[A(\DICE/)] - \Pr[A(\DUNCE/)]\rvert \\
  &\qquad + \lvert\Pr[A(\DUNCE/)] - \Pr[A(U_1)]\rvert \\
  &\leq \frac{(E + D)^2}{2^{192}}
          + \frac{D}{2^{192}} + D E \collisionbound
          + \binom{E}{2} \biggl(\collisionbound + \frac{1}{2^{192}}\biggr) \\
  &= \frac{(E + D)^2 + D + \binom{E}{2}}{2^{192}}
     + 64\lceil\ell/16\rceil^2 \frac{D E + \binom{E}{2}}{2^{212}}.
\end{align*}

\subsection{Extending the idealization to multiple instances}

Let $A$ be a random decision algorithm with access to encryption and
 decryption oracles for a \emph{collection} of deterministic
 authenticated ciphers.
Assume $A$ makes $E(n)$ queries to the $n^{\mathit{th}}$ encryption
 oracle and $D(n)$ queries to the $n^{\mathit{th}}$ decryption oracle.

In the foregoing analysis, the critical events---$C_{th}$, $F$,
 $C_{h'}$, $C_h$, and $C_t$---can be straightforwardly seen to sum
 over the instances independently.
For example, the event $C_{th}$ of a collision in \emph{any one} of
 the instances each with its own independent function $f$, $k_1$, and
 $k_2$ is the sum over all instances of $C_{th}$ for that instance
 separately.
Consequently, the multi-instance \DAE/ advantage against \DEUCE/ is at
 most the sum of the single-instance \DAE/ advantages:
%
\begin{align*}
  \Adv^{\muDAE}_{\DEUCE/}(A)
  &\leq \sum_n \frac{\bigl(E(n) + D(n)\bigr)^2 + D(n) + \binom{E(n)}{2}}
                    {2^{192}} \\
  &\qquad\qquad
          + 64\lceil\ell/16\rceil^2
            \frac{D(n) E(n) + \binom{E(n)}{2}}{2^{212}}.
\end{align*}

\subsection{Instantiating the idealized cipher}

Let $A$ be a random decision algorithm with access to encryption and
 decryption oracles for a \emph{collection} of deterministic
 authenticated ciphers.
Assume $A$ makes $E(n)$ queries to the $n^{\mathit{th}}$ encryption
 oracle and $D(n)$ queries to the $n^{\mathit{th}}$ decryption oracle.

\DEUCE/ is simply $\DAENCE/_1$ with a uniform random function $f$
 substituted for $\XSalsa20/_{k_0}$, so if $A$ can distinguish a
 collection of $\DAENCE/_1$ instances from a collection of \DEUCE/
 instances then it can be used in an algorithm $A'$ to distinguish
 a collection of $\XSalsa20/_{k_0}$ instances from a collection of $f$
 instances (recall $f\colon \{0,1\}^{256} \to \{0,1\}^{512}$ is a
 uniform random function).

If $\mathcal O(n,x)$ is an oracle for a collection of function
 instances indexed by $n$, define $A'(\mathcal O)$ to run $A$ with
 oracles for the following \DAE/ under independent uniform random keys
 $k_1 \concat k_2$ for the $n^{\mathit{th}}$ instance, decryption
 being defined the obvious way:
%
\begin{enumerate}
  \item $h := \Poly1305/^2_{k_1,k_2}(m)$
    \hfill (32-byte $h$)
  \item $t := \hat{\mathcal O}_{192}(n, h)$
    \hfill (24-byte $t$)
  \item $c := m \oplus \mathcal O_*(n, t)$
\end{enumerate}
%
The algorithm $A'$ performs
 $\sum_n (1 + \lceil\ell/64\rceil) \bigl(E(n) + D(n)\bigr)$ oracle
 queries, and spends the cost of computing $A$ plus the cost of
 evaluating $\Poly1305/^2$ and $\oplus$ on $\sum_n E(n) + D(n)$
 different $\ell$-byte inputs; and
%
\begin{multline*}
  \lvert\Pr[A(\DAENCE/_1)] - \Pr[A(\DEUCE/)]\rvert
   = \lvert\Pr[A'(\XSalsa20/_{k_0})] - \Pr[A'(f)]\rvert \\
   = \Adv^{\muPRF}_{\XSalsa20/}(A').
\end{multline*}

\subsection{Multiple users \vs\ multiple nonces under idealized key derivation}

For each user, let $F\colon \{0,1\}^{128} \to \{0,1\}^{512}$ be an
 independent uniform random key derivation function, which will be
 used to derive subkeys for $\DAENCE/_1$ from a nonce $n$ by
 $k_0 \concat k_1 \concat k_2 = F(n)$.
This is obviously equivalent to a single independent uniform random
 key derivation function $F'(u,n) \in \{0,1\}^{512}$ for the user $u$
 and nonce $n$, used for the same purpose---note that the inputs to
 $F'$ for two different users never collide even if they use the same
 nonce.

In other words, under an idealization of a key derivation function,
 there is no difference between multiple \emph{users} and multiple
 \emph{nonces}.
Define the nonce-based authenticated cipher \term{$\DAENCE/_*$} to be
 as \DAENCE/, but with a uniform random function $F$ in the place of
 $\Salsa20/_k$ for key derivation:
%
\begin{enumerate}
  \item $k_0 \concat k_1 \concat k_2 := F(n)$
    \hfill (32-byte $k_0$, 16-byte $k_1$, 16-byte $k_2$)
  \item $t \concat \_ := \HXSalsa20/_{k_0}(\Poly1305/^2_{k_1,k_2}(m))$
    \hfill (24-byte $t$)
  \item $c := m \oplus \bigl(\XSalsa20/_{k_0}(t \mathbin\| 0) \concat
      \XSalsa20/_{k_0}(t \mathbin\| 1) \concat \dotsb\bigr)$
\end{enumerate}

The multi-\emph{user} security of $\DAENCE/_*$, viewed as a collection
 of \emph{nonce-based} authenticated ciphers with independent key
 derivation functions $F$, is obviously the same as the
 multi-\emph{instance} security of $\DAENCE/_1$, viewed as a
 collection of \emph{deterministic} authenticated ciphers with
 independent keys $F'(u,n)$.

Thus, if $A_*$ is a random decision algorithm with oracle access to a
 collection of \emph{nonce-based} authenticated ciphers $D_*$, then
 there is a random decision algorithm $A$ with oracle access to a
 collection of \emph{deterministic} authenticated ciphers $D_1$ having
 exactly the same cost as $A_*$ such that
\[
  \Pr[A_*(D_*)] = \Pr[A(D_1)].
\]
This applies equally to the pathological ciphers $U_*$ and $U_1$ and
 to the $\DAENCE/_*$ and $\DAENCE/_1$, so
%
\begin{align*}
  \Adv^{\muNMRAE}_{\DAENCE/_*}(A_*)
  &= \lvert\Pr[A_*(\DAENCE/_*)] - \Pr[A_*(U_*)]\rvert \\
  &= \lvert\Pr[A(\DAENCE/_1)] - \Pr[A(U_1)]\rvert.
\end{align*}

\subsection{Instantiating the idealized key derivation}

Let $A_*$ be a random decision algorithm with access to oracles for a
 collection of nonce-based authenticated ciphers.
Assume $A_*$ makes $E(u,n)$ encryption queries under nonce $n$ to the
 $u^{\mathit{th}}$ user's cipher, and $D(u,n)$ decryption queries.

$\DAENCE/_*$ is simply $\DAENCE/$ with a uniform random function $F$
 substituted for $\Salsa20/_k$, so if $A_*$ can distinguish a
 collection of $\DAENCE/_*$ instances from a collection of $\DAENCE/$
 instances then it can be used in an algorithm to distinguish a
 collection of $\Salsa20/_k$ instances from a collection of $F$
 instances (recall $F\colon \{0,1\}^{128} \to \{0,1\}^{512}$ is a
 uniform random function).

If $\mathcal O(u,n)$ is an oracle for a collection of function
 instances indexed by $u$, define $A'_*(\mathcal O)$ to run $A_*$ with
 oracles for the following nonce-based authenticated cipher under
 nonce $n$, decryption being defined the obvious way:
%
\begin{enumerate}
  \item $k_0 \concat k_1 \concat k_2 := \mathcal O(u, n)$
    \hfill (32-byte $k_0$, 16-byte $k_1$, 16-byte $k_2$)
  \item $t \concat \_ := \HXSalsa20/_{k_0}(\Poly1305/^2_{k_1,k_2}(m))$
    \hfill (24-byte $t$)
  \item $c := m \oplus \bigl(\XSalsa20/_{k_0}(t \mathbin\| 0) \concat
      \XSalsa20/_{k_0}(t \mathbin\| 1) \concat \dotsb\bigr)$
\end{enumerate}
%
The algorithm $A'_*$ performs $\sum_{u,n} E(u,n) + D(u,n)$ oracle
 queries, and spends the cost of computing $A_*$ plus the cost of
 evaluating $\Poly1305/^2$, $\XSalsa20/$, \etc, on
 $\sum_{u,n} E(u,n) + D(u,n)$ different $\ell$-byte inputs; and
%
\begin{multline*}
  \lvert\Pr[A_*(\DAENCE/)] - \Pr[A_*(\DAENCE/_*)]\rvert
   = \lvert\Pr[A'_*(\Salsa20/_k)] - \Pr[A'_*(F)]\rvert \\
   = \Adv^{\muPRF}_{\Salsa20/}(A'_*).
\end{multline*}

\subsection{Tying the room together}

By stringing all the inequalities together, we complete the proof:
%
\begin{align*}
  \Adv^{\muNMRAE}_{\DAENCE/}(A_*)
  &= \lvert\Pr[A_*(\DAENCE/)] - \Pr[A_*(U)]\rvert \\
  &\leq \lvert\Pr[A_*(\DAENCE/)] - \Pr[A_*(\DAENCE/_*)]\rvert \\
  &\qquad + \lvert\Pr[A_*(\DAENCE/_*)] - \Pr[A_*(U)]\rvert \\
  &\leq \Adv^{\muPRF}_{\Salsa20/}(A'_*)
          + \lvert\Pr[A(\DAENCE/_1)] - \Pr[A(U_1)]\rvert \\
  &\leq \Adv^{\muPRF}_{\Salsa20/}(A'_*)
          + \lvert\Pr[A(\DAENCE/_1)] - \Pr[A(\DEUCE/)]\rvert \\
  &\qquad + \lvert\Pr[A(\DEUCE/)] - \Pr[A(U_1)]\rvert \\
  &\leq \Adv^{\muPRF}_{\Salsa20/}(A'_*)
          + \Adv^{\muPRF}_{\XSalsa20/}(A') \\
  &\qquad + \sum_{u,n}
              \frac{\bigl(E(u,n) + D(u,n)\bigr)^2 + D(u,n) + \binom{E(u,n)}{2}}
                   {2^{192}} \\
  &\qquad\qquad\qquad
          + 64\lceil\ell/16\rceil^2
            \frac{D(u,n) E(u,n) + \binom{E(u,n)}{2}}{2^{212}}.
\end{align*}

\section{Shaving IAQs---Infrequently Asked Questions}

\begin{description}
  \item[Why not just use the \CAESAR/ competition winner?]
    In our estimation, the \CAESAR/
     competition\footnote{\CAESAR/: Competition for Authenticated
        Encryption: Security, Applicability, and Robustness.
       \url{https://competitions.cr.yp.to/caesar.html}}
     was too broad and dragged on for too long, and in the end failed
     to gain the traction it needed.
    While \NaCl/, \IETF/ \ChaCha//\Poly1305/, and \AESGCM/ are
     ubiquitous today, with a variety of high-quality implementations
     available in many programming languages, the benefits of the
     \CAESAR/ winners do not seem to justify the engineering effort to
     make the novel cryptographic primitives as ubiquitous.
    In contrast, \DAENCE/ requires negligible engineering effort on
     top of \NaCl/ or \libsodium/.

  \item[Why bother with \Salsa20/?  Why not just \AEAD/ with \ChaCha/?]
    \hspace{\stretch{1}}\penalty-300
    The \NaCl/ \naclonetimeauth/ interface is widely available in many
     languages, particularly via TweetNaCl.
    It is easy to implement \DAENCE/ without associated data in terms
     of this interface; it is not easy to use it to handle associated
     data.

    On the other hand, \ChaCha//\Poly1305/ is becoming more
     prevalent following \IETF/ standardization in \TLS/, and \AEAD/s
     are useful in their own right.
    So we also provide a nonce-misuse-resistant \AEAD/ that can be
     easily implemented in terms of \ChaCha/ and \Poly1305/ as found
     in \libsodium/ and likely other libraries.

  \item[Why a nonce?]
    Instead of having a nonce, we could \emph{start} with a 512-bit
     key $k_0 \concat k_1 \concat k_2$ (32-byte $k_0$, 16-byte $k_1$,
     16-byte $k_2$), and derive the 512-bit subkey
    \[
      k'_0 \concat k'_1 \concat k'_2
      = \XSalsa20/_{k_0}(\Poly1305/^2_{k_1,k_2}(a)),
    \]
     where $a$ is the associated data; then use $k'_0$, $k'_1$, and
     $k'_2$ where we currently use $k_0$, $k_1$, and $k_2$ in the
     definition of \DAENCE/.

    This is \emph{effectively} almost like a nonce---there's a
     slightly higher chance of collision, but under nonce reuse,
     the security simply reduces to deterministic authenticated
     encryption, so it's not the end of the world.
    This would simplify the conceptual model by obviating the need for
     any `nonce' input at all, at the minor cost of requiring a
     512-bit key instead of a 256-bit key---and it would enable the
     use of associated data in \NaCl/.

    But\dots it would require adjusting the analysis above, which I
     will leave for another day.
\end{description}

\noindent
We do not anticipate that \DAENCE/ will replace \ChaCha//\Poly1305/ in
 major protocols such as \TLS/ designed by world-class cryptographers,
 which can easily take advantage of a message number guaranteed not to
 repeat.
We do hope that \DAENCE/ will find its way into the repertoire of
 general-purpose application engineers who need to store messages safe
 from eavesdropping and forgery in diverse software
 environments---%
 \emph{without} auditing the software stack all the way down to
  machine instructions to ensure their use of \AESGCMSIV/ is safe from
  timing side channels,
 \emph{without} adopting unusual cryptographic primitives or an
  entirely new cryptography library, and
 \emph{without} teetering on the brink of catastrophe from nonce reuse
  in \naclsecretbox/ or \AESGCM/.

\bibliography{daence}

\appendix

\section{Reference implementation}

\lstinputlisting[basicstyle=\small\ttfamily,tabsize=4]{testvector.c}

\section{Test vectors}

\lstinputlisting[basicstyle=\small\ttfamily]{testvector.out}

\end{document}
